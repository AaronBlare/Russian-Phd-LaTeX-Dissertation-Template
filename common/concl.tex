%% Согласно ГОСТ Р 7.0.11-2011:
%% 5.3.3 В заключении диссертации излагают итоги выполненного исследования, рекомендации, перспективы дальнейшей разработки темы.
%% 9.2.3 В заключении автореферата диссертации излагают итоги данного исследования, рекомендации и перспективы дальнейшей разработки темы.
\begin{enumerate}
  \item Разработан метод поиска связанных с возрастом специфичных для пола биомаркеров, включающий также раздельный поиск связанных с возрастом и специфичных для пола биомаркеров. Он основан на проведение корреляционного анализа и метаанализа множества наборов данных.
  \item Разработан алгоритм поиска биомаркеров с возрастной вариабельностью, связанной с полом, предполагающий тестирование на гетероскедастичность и проведение метаанализа множества наборов данных. Описаны основные сценарии поведения половых различий в возрастной вариабельности уровня метилирования ДНК.
  \item Разработан программный комплекс, реализующий указанные методы и позволяющий осуществлять полный цикл работы алгоритма от обработки входных данных до вывода результирующих биомаркеров и визуализации. Метод доступен для использования в виде программного пакета для языка Python в репозитории Python Package Index (https://pypi.org/project/pydnameth).
  \item Разработан метод поиска митохондриально-ядерных взаимодействий, специфичных для популяций, проживающих в определённых климатических условиях. Он основан на построении серии последовательных моделей случайного леса с увеличивающимся количеством признаков. Кроме того, ввиду большой размерности входных данных, разработан алгоритм её сокращения путём усреднения информации о генетических вариациях внутри каждого гена.
  \item Разработан программный комплекс, реализующий указанные алгоритмы и позволяющий осуществлять полный цикл работы алгоритма: обработка входных данных, уменьшение размерности входных данных, реализация алгоритма, получение результатов.
  \item Для всех разработанных методов проведена экспериментальная проверка работоспособности на реальных биологических данных, показана корректность получаемых результатов, приведено обоснование с биологической точки зрения.
\end{enumerate}
