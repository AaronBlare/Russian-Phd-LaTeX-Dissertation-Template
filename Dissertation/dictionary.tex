\chapter*{Словарь терминов}             % Заголовок
\addcontentsline{toc}{chapter}{Словарь терминов}  % Добавляем его в оглавление

\textbf{Соматические клетки} : Все клетки, составляющее тело многоклеточного организма, за исключением гамет

\textbf{Эукариоты} : Живые организмы, клетки которых содержат ядро

\textbf{Метилирование ДНК} : Присоединение метильной группы к последовательности ДНК

\textbf{Гипометилирование} : Снижение уровня метилирования ДНК

\textbf{Гиперметилирование} : Повышение уровня метилирования ДНК

\textbf{Секвенирование ДНК} : Определение нуклеотидной последовательности ДНК

\textbf{Гаплогруппа} : Группа субъектов, имеющих общего предка с мутацией, унаследованной всеми потомками

\textbf{Эпигенетика} : Раздел биологии, изучающий наследственные изменения фенотипа, не связанные с изменением последовательности ДНК

\textbf{Сайт CpG} : Область ДНК, в которой за нуклеотидом C (цитозин) следует нуклеотид G (гуанин)

\textbf{Однонуклеотидный полиморфизм} : Замена одного нуклеотида в определённом положении в геноме у представителей одного виде, присутствующий в большой части популяции 

\textbf{Метаанализ} : Статистический анализ, объединяющий результаты нескольких научных исследований

\textbf{P-значение} : Вероятность получения для заданной вероятностной модели такое же или более экстремальное значение вычисляемой статистики, по сравнению с фактически наблюдаемыми результатами, при условии, что нулевая гипотеза верна.

