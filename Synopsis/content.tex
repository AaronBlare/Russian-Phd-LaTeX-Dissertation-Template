\pdfbookmark{Общая характеристика работы}{characteristic}             % Закладка pdf
\section*{Общая характеристика работы}

\newcommand{\actuality}{\pdfbookmark[1]{Актуальность}{actuality}\underline{\textbf{\actualityTXT}}}
\newcommand{\progress}{\pdfbookmark[1]{Разработанность темы}{progress}\underline{\textbf{\progressTXT}}}
\newcommand{\aim}{\pdfbookmark[1]{Цели}{aim}\underline{{\textbf\aimTXT}}}
\newcommand{\tasks}{\pdfbookmark[1]{Задачи}{tasks}\underline{\textbf{\tasksTXT}}}
\newcommand{\aimtasks}{\pdfbookmark[1]{Цели и задачи}{aimtasks}\aimtasksTXT}
\newcommand{\novelty}{\pdfbookmark[1]{Научная новизна}{novelty}\underline{\textbf{\noveltyTXT}}}
\newcommand{\influence}{\pdfbookmark[1]{Практическая значимость}{influence}\underline{\textbf{\influenceTXT}}}
\newcommand{\methods}{\pdfbookmark[1]{Методология и методы исследования}{methods}\underline{\textbf{\methodsTXT}}}
\newcommand{\defpositions}{\pdfbookmark[1]{Положения, выносимые на защиту}{defpositions}\underline{\textbf{\defpositionsTXT}}}
\newcommand{\reliability}{\pdfbookmark[1]{Достоверность}{reliability}\underline{\textbf{\reliabilityTXT}}}
\newcommand{\probation}{\pdfbookmark[1]{Апробация}{probation}\underline{\textbf{\probationTXT}}}
\newcommand{\contribution}{\pdfbookmark[1]{Личный вклад}{contribution}\underline{\textbf{\contributionTXT}}}
\newcommand{\publications}{\pdfbookmark[1]{Публикации}{publications}\underline{\textbf{\publicationsTXT}}}


{\actuality}
Развитие современных технологий в биологии и биомедицине привело к накоплению большого количества разнородных данных, характеризующих организм как на генетическом, так и эпигенетическом уровне. Для получения новых знаний о генотипических и фенотипических особенностях организма человека, необходимо проводить многомерный анализ рассматриваемых данных и интерпретировать получаемые результаты.  Поскольку с увеличением объёма данных проведение анализа вручную становится чрезвычайно затруднительным, требуется разработка соответствующих вычислительных методов. 

Одним из наиболее актуальных направлений современной биологической науки является эпигенетика, изучающая изменение активности экспрессии генов без изменения нуклеотидной последовательности ДНК. Известно, что эпигенетические факторы могут оказывать значительное влияние на здоровье человека, а также играть важную роль в процессах старения. Наиболее известным эпигенетическим механизмом является метилирование ДНК (присоединение метильной группы к последовательности ДНК), аномальное изменение уровня которого сопровождает развитие многих заболеваний. Несмотря на то, что существуют исследования в области взаимосвязи уровня метилирования и биологического возраста человека, остаётся актуальным вопрос о связи с биологическим полом, слабо изученного ввиду большого числа зависимостей и сложности анализа. 

Исследование мутаций в геноме человека также представляет собой актуальную задачу в современной генетике. Разработка и совершенствование технологий секвенирования генетических данных приводит к удешевлению процесса получения генетической информации, что влечёт за собой необходимость совершенствования алгоритмов обработки и анализа больших данных, либо методов понижения размерности задачи. Стоит отметить, что генетическую информацию несёт в себе не только ядерная ДНК, содержащаяся в ядрах всех клеток живого организма, но и митохондриальная ДНК, располагающаяся в клеточных органеллах -- митохондриях. Существующие исследования ввиду большой размерности входных данных (3 млрд пар нуклеотидов) ограничиваются рассмотрением только одного из типов геномов, опуская митохондриально-ядерные взаимодействия. Совместное изучение однонуклеотидных полиморфизмов (мутаций) в митохондриальной ДНК и в связанной с митохондриальными функциями ядерной ДНК даёт возможность глубже исследовать, в частности, процессы адаптации к различным природным условиям у популяций, проживающих в разных климатических регионах.

В соответствии с паспортом специальности 05.13.18 <<Математическое моделирование, численные методы и комплексы программ>>, данная диссертация относится к следующим областям исследований: <<3. Разработка, обоснование и тестирование эффективных вычислительных методов с применением современных компьютерных технологий>>, <<5. Комплексные исследования научных и технических проблем с применением современной технологии математического моделирования и вычислительного эксперимента>>, <<7. Разработка новых математических методов и алгоритмов интерпретации натурного эксперимента на основе его математической модели>>.

{\aim} работы является разработка и программная реализация методов статистического анализа экспериментальных генетических и эпигенетических данных для оценки физиологического состояния человека.

Для достижения поставленной цели последовательно решаются следующие основные {\tasks} диссертационной работы:
\begin{enumerate}[beginpenalty=10000]
	\item Разработка и программная реализация метода поиска связанных с возрастом специфичных для пола биомаркеров на основе анализа экспериментальных генетических данных.
	\item Разработка и программная реализация метода поиска биомаркеров с возрастной вариабельностью, связанной с полом, на основе анализа экспериментальных генетических данных. 
	\item Разработка и программная реализация метода анализа экспериментальных митохондриально-ядерных генетических данных для оценки региональных различий адаптации к климатическим условиям. 
\end{enumerate}

{\novelty} В работе получены следующие новые научные результаты:
\begin{enumerate}[beginpenalty=10000] 
	\item Впервые разработан и реализован метод для проведения поиска связанных с возрастом специфичных для пола биомаркеров, основанный на построении регрессионной модели и проведении статистического мета-анализа. Новизна метода заключается в одновременном учёте зависимости уровня метилирования как от возраста, так и от пола.
	\item Впервые разработан и реализован метод для проведения поиска биомаркеров с возрастной вариабельностью, связанной с полом, основанный на построении регрессионной модели, тестировании на гетероскедастичность и проведении статистического мета-анализа. Новизна метода заключается в одновременном учёте зависимости уровня метилирования как от возраста, так и от пола. 
	\item Впервые разработан и реализован метод анализа экспериментальных митохондриально-ядерных генетических данных для оценки региональных различий адаптации к климатическим условиям, основанный на построении модели случайного леса. Новизна метода заключается в одновременном рассмотрении митохондриальных и ядерных однонуклеотидных полиморфизмов.
\end{enumerate}

{\methods} В работе используются методы машинного обучения, теории вероятностей и математической статистики.

{\defpositions}
\begin{enumerate}[beginpenalty=10000]
	\item Метод анализа эпигенетических данных для поиска связанных с возрастом специфичных для пола биомаркеров.
	\item Метод анализа эпигенетических данных для поиска биомаркеров с возрастной вариабельностью, связанной с полом.
	\item Метод анализа экспериментальных митохондриально-ядерных генетических данных для оценки региональных различий адаптации к климатическим условиям.
\end{enumerate}

В разделе \textit{Научная новизна} обозначены основные отличия и преимущества вышеуказанных методов.

{\reliability} научных положений, выводов и практических рекомендаций, полученных в диссертации, обеспечивается корректным обоснованием постановок задач, точной формулировкой критериев,  подтверждается результатами вычислительных экспериментов по использованию предложенных в диссертации методов и алгоритмов.

{\probation} Основные результаты настоящей диссертационной работы докладывались на следующих научных конференциях:
\begin{itemize}[beginpenalty=10000]
	\item 72-я Всероссийская с международным участием школа-конференция молодых учёных <<Биосистемы: организация, поведение, управление>>. 2019, ННГУ им. Н.И. Лобачевского, Нижний Новгород.
	\item The 9th International Scientific Conference on Physics and Control <<PhysCon2019>>. 2019, Университет Иннополис, Иннополис.
	\item XXIV научная конференция по радиофизике, посвящённая 75-летию радиофизического факультета. 2020, ННГУ им. Н.И. Лобачевского, Нижний Новгород.
	\item 73-я Всероссийская с международным участием школа-конференция молодых учёных <<Биосистемы: организация, поведение, управление>>. 2020, ННГУ им. Н.И. Лобачевского, Нижний Новгород.
\end{itemize}

{\contribution} Решение задач диссертационного исследования, разработка и реализация метода анализа экспериментальных митохондриально-ядерных генетических данных для оценки региональных различий адаптации к климатическим условиям принадлежат автору лично. Разработка реализаций для методов анализа эпигенетических данных для поиска биомаркеров с возрастной вариабельностью, связанной с полом, и анализа экспериментальных митохондриально-ядерных генетических данных для оценки региональных различий адаптации к климатическим условиям, была выполнена в соавторстве с Юсиповым И.И.

\ifnumequal{\value{bibliosel}}{0}
{%%% Встроенная реализация с загрузкой файла через движок bibtex8.
    {\publications} Основные результаты по теме диссертации изложены
    в~XX~печатных изданиях,
    X из которых изданы в журналах, рекомендованных ВАК,
    X "--- в тезисах докладов.
}%
{%%% Реализация пакетом biblatex через движок biber
    \begin{refsection}[bl-author, bl-registered]
        % Это refsection=1.
        % Процитированные здесь работы:
        %  * подсчитываются, для автоматического составления фразы "Основные результаты ..."
        %  * попадают в авторскую библиографию, при usefootcite==0 и стиле `\insertbiblioauthor` или `\insertbiblioauthorgrouped`
        %  * нумеруются там в зависимости от порядка команд `\printbibliography` в этом разделе.
        %  * при использовании `\insertbiblioauthorgrouped`, порядок команд `\printbibliography` в нём должен быть тем же (см. biblio/biblatex.tex)
        %
        % Невидимый библиографический список для подсчёта количества публикаций:
        \printbibliography[heading=nobibheading, section=1, env=countauthorvak,          keyword=biblioauthorvak]%
        \printbibliography[heading=nobibheading, section=1, env=countauthorwos,          keyword=biblioauthorwos]%
        \printbibliography[heading=nobibheading, section=1, env=countauthorscopus,       keyword=biblioauthorscopus]%
        \printbibliography[heading=nobibheading, section=1, env=countauthorconf,         keyword=biblioauthorconf]%
        \printbibliography[heading=nobibheading, section=1, env=countauthorother,        keyword=biblioauthorother]%
        \printbibliography[heading=nobibheading, section=1, env=countregistered,         keyword=biblioregistered]%
        \printbibliography[heading=nobibheading, section=1, env=countauthorpatent,       keyword=biblioauthorpatent]%
        \printbibliography[heading=nobibheading, section=1, env=countauthorprogram,      keyword=biblioauthorprogram]%
        \printbibliography[heading=nobibheading, section=1, env=countauthor,             keyword=biblioauthor]%
        \printbibliography[heading=nobibheading, section=1, env=countauthorvakscopuswos, filter=vakscopuswos]%
        \printbibliography[heading=nobibheading, section=1, env=countauthorscopuswos,    filter=scopuswos]%
        %
        \nocite{*}%
        %
        {\publications} Основные результаты по теме диссертации изложены в~\arabic{citeauthor}~печатных изданиях,
        \arabic{citeauthorvak} из которых изданы в журналах, рекомендованных ВАК\sloppy%
        \ifnum \value{citeauthorscopuswos}>0%
            , \arabic{citeauthorscopuswos} "--- в~периодических научных журналах, индексируемых Web of~Science и Scopus\sloppy%
        \fi%
        \ifnum \value{citeauthorconf}>0%
            , \arabic{citeauthorconf} "--- в~тезисах докладов.
        \else%
            .
        \fi%
        \ifnum \value{citeregistered}=1%
            \ifnum \value{citeauthorpatent}=1%
                Зарегистрирован \arabic{citeauthorpatent} патент.
            \fi%
            \ifnum \value{citeauthorprogram}=1%
                Зарегистрирована \arabic{citeauthorprogram} программа для ЭВМ.
            \fi%
        \fi%
        \ifnum \value{citeregistered}>1%
            Зарегистрированы\ %
            \ifnum \value{citeauthorpatent}>0%
            \formbytotal{citeauthorpatent}{патент}{}{а}{}\sloppy%
            \ifnum \value{citeauthorprogram}=0 . \else \ и~\fi%
            \fi%
            \ifnum \value{citeauthorprogram}>0%
            \formbytotal{citeauthorprogram}{программ}{а}{ы}{} для ЭВМ.
            \fi%
        \fi%
        % К публикациям, в которых излагаются основные научные результаты диссертации на соискание учёной
        % степени, в рецензируемых изданиях приравниваются патенты на изобретения, патенты (свидетельства) на
        % полезную модель, патенты на промышленный образец, патенты на селекционные достижения, свидетельства
        % на программу для электронных вычислительных машин, базу данных, топологию интегральных микросхем,
        % зарегистрированные в установленном порядке.(в ред. Постановления Правительства РФ от 21.04.2016 N 335)
    \end{refsection}%
    \begin{refsection}[bl-author, bl-registered]
        % Это refsection=2.
        % Процитированные здесь работы:
        %  * попадают в авторскую библиографию, при usefootcite==0 и стиле `\insertbiblioauthorimportant`.
        %  * ни на что не влияют в противном случае
        \nocite{prog1}%program
        \nocite{conf1}%conf
        \nocite{conf2}%conf
        \nocite{conf3}%conf
        \nocite{conf4}%conf
    \end{refsection}%
        %
        % Всё, что вне этих двух refsection, это refsection=0,
        %  * для диссертации - это нормальные ссылки, попадающие в обычную библиографию
        %  * для автореферата:
        %     * при usefootcite==0, ссылка корректно сработает только для источника из `external.bib`. Для своих работ --- напечатает "[0]" (и даже Warning не вылезет).
        %     * при usefootcite==1, ссылка сработает нормально. В авторской библиографии будут только процитированные в refsection=0 работы.
}


 % Характеристика работы по структуре во введении и в автореферате не отличается (ГОСТ Р 7.0.11, пункты 5.3.1 и 9.2.1), потому её загружаем из одного и того же внешнего файла, предварительно задав форму выделения некоторым параметрам

%Диссертационная работа была выполнена при поддержке грантов \dots

%\underline{\textbf{Объем и структура работы.}} Диссертация состоит из~введения,
%четырех глав, заключения и~приложения. Полный объем диссертации
%\textbf{ХХХ}~страниц текста с~\textbf{ХХ}~рисунками и~5~таблицами. Список
%литературы содержит \textbf{ХХX}~наименование.

\pdfbookmark{Содержание работы}{description}                          % Закладка pdf
\section*{Содержание работы}
Во \textbf{введении} обосновывается актуальность исследований, проводимых в~рамках данной диссертационной работы, формулируется цель, ставятся задачи, излагается научная новизна и практическая значимость представляемой работы.

В \textbf{первой главе} приводится обзор предметной области и результаты существующих исследований, посвящённых поиску различных генетических и эпигенетических биомаркеров. На основе данных метилирования приводится описание работ, анализирующих их связь с возрастом, либо полом. Информация об однонуклеотидных полиморфизмах используется для определения адаптационных зависимостей, связанных с определёнными климатическими условиями. Кроме того, в первой главе приводится обзор ряда методов машинного обучения, теории вероятностей и математической статистики, используемых для реализации предлагаемых в диссертационной работе алгоритмов и методов: регрессионный анализ, обучение с учителем, в частности, построение случайного леса, метаанализ.

В \textbf{разделе 1.1} приведено описание генетических и эпигенетических факторов. 



\underline{\textbf{Первая глава}} посвящена \dots

картинку можно добавить так:
\begin{figure}[ht]
    \centerfloat{
        \hfill
        \subcaptionbox{\LaTeX}{%
            \includegraphics[scale=0.27]{latex}}
        \hfill
        \subcaptionbox{Knuth}{%
            \includegraphics[width=0.25\linewidth]{knuth1}}
        \hfill
    }
    \caption{Подпись к картинке.}\label{fig:latex}
\end{figure}

Формулы в строку без номера добавляются так:
\[
  \lambda_{T_s} = K_x\frac{d{x}}{d{T_s}}, \qquad
  \lambda_{q_s} = K_x\frac{d{x}}{d{q_s}},
\]

\underline{\textbf{Вторая глава}} посвящена исследованию

\underline{\textbf{Третья глава}} посвящена исследованию

Можно сослаться на свои работы в автореферате. Для этого в файле
\verb!Synopsis/setup.tex! необходимо присвоить положительное значение
счётчику \verb!\setcounter{usefootcite}{1}!. В таком случае ссылки на
работы других авторов будут подстрочными.
Изложенные в третьей главе результаты опубликованы в~\cite{vakbib1, vakbib2}.
Использование подстрочных ссылок внутри таблиц может вызывать проблемы.

В \underline{\textbf{четвертой главе}} приведено описание

\FloatBarrier
\pdfbookmark{Заключение}{conclusion}                                  % Закладка pdf
В \underline{\textbf{заключении}} приведены основные результаты работы, которые заключаются в следующем:
%% Согласно ГОСТ Р 7.0.11-2011:
%% 5.3.3 В заключении диссертации излагают итоги выполненного исследования, рекомендации, перспективы дальнейшей разработки темы.
%% 9.2.3 В заключении автореферата диссертации излагают итоги данного исследования, рекомендации и перспективы дальнейшей разработки темы.
\begin{enumerate}
  \item Разработан метод поиска связанных с возрастом специфичных для пола биомаркеров, включающий также раздельный поиск связанных с возрастом и специфичных для пола биомаркеров. Он основан на проведение корреляционного анализа и метаанализа множества наборов данных.
  \item Разработан алгоритм поиска биомаркеров с возрастной вариабельностью, связанной с полом, предполагающий тестирование на гетероскедастичность и проведение метаанализа множества наборов данных. Описаны основные сценарии поведения половых различий в возрастной вариабельности уровня метилирования ДНК.
  \item Разработан программный комплекс, реализующий указанные методы и позволяющий осуществлять полный цикл работы алгоритма от обработки входных данных до вывода результирующих биомаркеров и визуализации. Метод доступен для использования в виде программного пакета для языка Python в репозитории Python Package Index (https://pypi.org/project/pydnameth).
  \item Разработан метод поиска митохондриально-ядерных взаимодействий, специфичных для популяций, проживающих в определённых климатических условиях. Он основан на построении серии последовательных моделей случайного леса с увеличивающимся количеством признаков. Кроме того, ввиду большой размерности входных данных, разработан алгоритм её сокращения путём усреднения информации о генетических вариациях внутри каждого гена.
  \item Разработан программный комплекс, реализующий указанные алгоритмы и позволяющий осуществлять полный цикл работы алгоритма: обработка входных данных, уменьшение размерности входных данных, реализация алгоритма, получение результатов.
  \item Для всех разработанных методов проведена экспериментальная проверка работоспособности на реальных биологических данных, показана корректность получаемых результатов, приведено обоснование с биологической точки зрения.
\end{enumerate}


\pdfbookmark{Литература}{bibliography}                                % Закладка pdf
При использовании пакета \verb!biblatex! список публикаций автора по теме
диссертации формируется в разделе <<\publications>>\ файла
\verb!common/characteristic.tex!  при помощи команды \verb!\nocite!

\ifdefmacro{\microtypesetup}{\microtypesetup{protrusion=false}}{} % не рекомендуется применять пакет микротипографики к автоматически генерируемому списку литературы
\urlstyle{rm}                               % ссылки URL обычным шрифтом
\ifnumequal{\value{bibliosel}}{0}{% Встроенная реализация с загрузкой файла через движок bibtex8
  \renewcommand{\bibname}{\large \bibtitleauthor}
  \nocite{*}
  \insertbiblioauthor           % Подключаем Bib-базы
  %\insertbiblioexternal   % !!! bibtex не умеет работать с несколькими библиографиями !!!
}{% Реализация пакетом biblatex через движок biber
  % Цитирования.
  %  * Порядок перечисления определяет порядок в библиографии (только внутри подраздела, если `\insertbiblioauthorgrouped`).
  %  * Если не соблюдать порядок "как для \printbibliography", нумерация в `\insertbiblioauthor` будет кривой.
  %  * Если цитировать каждый источник отдельной командой --- найти некоторые ошибки будет проще.
  %
  %% authorprogram
  \nocite{prog1}%
  %
  %% authorconf
  \nocite{conf1}%
  \nocite{conf2}%
  \nocite{conf3}%
  \nocite{conf4}%

  \ifnumgreater{\value{usefootcite}}{0}{
    \begin{refcontext}[labelprefix={}]
      \ifnum \value{bibgrouped}>0
        \insertbiblioauthorgrouped    % Вывод всех работ автора, сгруппированных по источникам
      \else
        \insertbiblioauthor      % Вывод всех работ автора
      \fi
    \end{refcontext}
  }{
  \ifnum \totvalue{citeexternal}>0
    \begin{refcontext}[labelprefix=A]
      \ifnum \value{bibgrouped}>0
        \insertbiblioauthorgrouped    % Вывод всех работ автора, сгруппированных по источникам
      \else
        \insertbiblioauthor      % Вывод всех работ автора
      \fi
    \end{refcontext}
  \else
    \ifnum \value{bibgrouped}>0
      \insertbiblioauthorgrouped    % Вывод всех работ автора, сгруппированных по источникам
    \else
      \insertbiblioauthor      % Вывод всех работ автора
    \fi
  \fi
  %  \insertbiblioauthorimportant  % Вывод наиболее значимых работ автора (определяется в файле characteristic во второй section)
  \begin{refcontext}[labelprefix={}]
      \insertbiblioexternal            % Вывод списка литературы, на которую ссылались в тексте автореферата
  \end{refcontext}
  % Невидимый библиографический список для подсчёта количества внешних публикаций
  % Используется, чтобы убрать приставку "А" у работ автора, если в автореферате нет
  % цитирований внешних источников.
  \printbibliography[heading=nobibheading, section=0, env=countexternal, keyword=biblioexternal, resetnumbers=true]%
  }
}
\ifdefmacro{\microtypesetup}{\microtypesetup{protrusion=true}}{}
\urlstyle{tt}                               % возвращаем установки шрифта ссылок URL
