\chapter{Обзор предметной области}\label{ch:ch1}

\section{Генетические и эпигенетические факторы}\label{sec:ch1/sec1}

\subsection{Виды и структура ДНК}\label{subsec:ch1/sec1/subsec1}

Дезоксирибонуклеиновая кислота (ДНК) обеспечивает реализацию и передачу из поколения в поколение генетической информации, касающейся развития, функционирования, роста и размножения живых организмов. Такие основные понятия современной генетики, как наследственность и изменчивость, связаны с молекулами ДНК. Наследственность определяют как способность живых организмов передавать свои признаки потомству, а изменчивость --- как способность приобретать отличия, обеспечивая тем самым биоразнообразие \autocite{rieger2012glossary}. 

ДНК представляет собой макромолекулу, состоящую из двух цепей нуклеотидов \autocite{brooker2004genetics,lesk2017introduction}. Эти цепи, закручиваясь друг вокруг друга, при помощи водородных связей образуют двойную спираль. Две цепи ДНК состоят из нуклеотидов, каждый из которых состоит из фосфатной группы, сахара (дезоксирибозы) и одного из четырёх азотистых оснований --- цитозин (C), гуанин (G), аденин (A) или тимин (T) \autocite{alberts2008molecular,berg2010biochemistry}. Каждое из оснований на одной цепи ДНК связано с определённым основанием на второй цепи по правилу комплементарности: цитозин связывается только с гуанином, аденин --- с тимином \autocite{tropp2012molecular}. Это означает, что вся информация, которую несёт одна цепь, дублируется на второй цепи. Такое специфическое обратимое взаимодействие имеет крайне важное значение для всех функций ДНК в живых организмах \autocite{alberts2017molecular}. 

Внутри ядер клеток эукариот располагается ядерная ДНК, структурированная в длинные структуры --- хромосомы. В ядрах клеток организма содержится человека 46 индивидуальных хромосом. Ядерная ДНК может существовать в нескольких копиях, число которых определяет термин <<плоидность>> \autocite{brown2002genomes,alberts2017molecular}. Соматические клетки человека диплоидны, то есть содержит две копии каждой хромосомы (всего 23 пары). Наследование ядерной ДНК является биродительским, то есть каждая из двух копий генома наследуется от одного из родителей --- матери или отца. 

Генетическая информация хранится в генах, а совокупность всех генов в организме называется его генотипом. Ген --- это единица наследственности и область ДНК, которая влияет на определённые характеристики организма. Ядерная ДНК человека содержит 20000--~25000 генов, однако, более 98\% генома не кодируют генетическую информацию \autocite{Wolfsberg2001}. Распространённой формой некодирующей ДНК у людей являются псевдогены, которые представляют собой копии генов, утративших функциональность в результате мутации \autocite{Harrison2002}.

В клетках эукариот ДНК располагается не только в ядрах, но и в митохондриях, клеточных органеллах, окисляющих органические соединения для преобразования высвобождающейся энергии в аденозинтрифосфат (АТФ). Митохондриальная ДНК была первой секвенированной частью генома, она состоит из 16569 пар оснований и кодирует 13 белков \autocite{Anderson1981}, некодирующие участки отсутствуют. Она в большей степени подвержена мутациям в сравнении с ядерной ДНК. Митохондриальная ДНК представляет собой двухцепочечную кольцевую молекулу ДНК, организованную в хромосому. Митохондриальная ДНК кодирует большинство белков, необходимых митохондриям, однако, часть их них кодируется ядерной ДНК. В свою очередь, ядерная ДНК кодирует все белки, требуемые клетке. Одна митохондрия содержит несколько десятков копий митохондриальной ДНК. Поскольку в одной клетке может находиться более 100 митохондрий, общее число копий митохондриальной ДНК превышает 1000 на одну клетку \autocite{lodish2012molecular}. 

Митохондриальная ДНК наследуется по материнской линии, поскольку цитоплазма в зиготу поступает только от яйцеклетки. Следовательно, заболевания, связанные с митохондриальной ДНК, передаются по наследству от матери \autocite{cooper2004cell}. Самый последний общий предок по материнской линии всех современных людей называют Митохондриальной Евой. Группа схожих гаплотипов (унаследованных от одного родителя аллелей), имеющих общего предка с определённой мутацией, называется гаплогруппой. Гаплогруппы относятся к одной линии наследования \autocite{cox2016biogeography}. То есть членство любого человека в гаплогруппе зависит от относительно небольшой части генетического материала, которым обладает этот человек. Каждая гаплогруппа происходит от одной предыдущей гаплогруппы и остаётся ее частью. Таким образом, любая родственная группа гаплогрупп моделируется как вложенная иерархия, в которой каждый набор (гаплогруппа) является подмножеством одного более широкого набора \autocite{Arora2015}. На вершине данной иерархии и располагается Митохондриальная Ева. Гаплогруппы обычно обозначаются заглавной буквой алфавита, а уточнения состоят из дополнительных комбинаций цифр и букв, например, $A \rightarrow A1 \rightarrow A1a$.

Изучением генов, генетической изменчивости и наследственности живых организмов занимается генетика \autocite{griffiths2000introduction}. Современная генетика вышла за рамки наследования и занимается также изучением функций и поведения генов в масштабах клетки, организма и популяции. Генетика дала начало ряду подразделов, включая молекулярную генетику, эпигенетику и популяционную генетику.

\subsection{Метилирование ДНК}\label{subsec:ch1/sec1/subsec2}

Фенотип организма характеризует физическую форму, структуру организма, его процессы развития, биохимические и физиологические свойства, поведение и многое другое. Фенотип определяется двумя основными факторами: генотип организма (совокупность всего генетического кода) и влиянием факторов окружающей среды. Оба фактора могут взаимодействовать, дополнительно влияя на фенотип \autocite{Dawkins2010}. В биологии эпигенетика изучает наследственные изменения фенотипа, не связанные с изменением последовательности ДНК \autocite{Dupont2009}. Эпигенетика чаще всего исследует изменения, которые влияют на активность и экспрессию генов, но этот термин также может использоваться для описания любых наследственных фенотипических изменений. Такое воздействие на клеточные и физиологические фенотипические признаки может быть результатом как внешних факторов, так и факторов окружающей среды или быть частью нормального развития. Эпигенетика также относится к функционально значимым изменениям в геноме, которые не связаны с изменением нуклеотидной последовательности. Наиболее известным примером механизма, вызывающего такие изменения, является метилирование ДНК, которое изменяет экспрессию генов без изменения последовательности ДНК \autocite{Bird2007}. Экспрессия генов может контролироваться действием специальных белков-репрессоров, прикрепляющихся к сайленсерам ДНК --- регионам, связывание с которыми приводит к снижению или подавлению синтеза РНК \autocite{Pang2020}.

Метилирование ДНК --- это биологический процесс, с помощью которого к молекуле ДНК добавляются метильные группы. Метилирование может изменить активность сегмента ДНК без изменения последовательности --- находясь в промоторе гена (стартовый регион гена для начала транскрипции), оно обычно подавляет транскрипцию гена. Метилировано может быть только одно из четырёх оснований ДНК --- цитозин, располагающийся в динуклеотиде CG (положение нуклеотидов в цепи ДНК, при котором в линейной последовательности за нуклеотидом C следует нуклеотид G) \autocite{BuckKoehntop2013, Kelsey2013}. Обычно их называют сайтами CpG.

У млекопитающих существуют регионы ДНК, богатые на CpG сайты --- эти области называются островами CpG \autocite{Bird1986}. Они определяются как регионы длиной более 200 bp (спаренных оснований), с содержанием динуклеотидов CG более $50~\%$ и отношением ожидаемого количества CpG сайтов к реальному больше $0.6$ \autocite{GardinerGarden1987}. Пример острова CpG приведён на рисунке~\cref{fig:CpG_Island}. В геноме человека имеется около 25000 островов CpG, $75~\%$  которых имеют длину менее 850 пар оснований \autocite{Lander2001}. Они являются основными регуляторными единицами, и около $50~\%$ островков CpG расположены в областях промоторов генов. В свою очередь, около $60-70~\%$ генов человека имеют остров CpG в своей промоторной области \autocite{Illingworth2010, Saxonov2006}.

\begin{figure}[ht]
	\centerfloat{
		\includegraphics[scale=3.0]{island.jpg}
	}
	\caption{Пример острова CpG (CpG island), региона с самой высокой плотностью размещения сайтов CpG в промоторной области гена. По обе стороны от острова CpG располагаются побережья (CpG island shores), регионы с меньшей плотностью размещения сайтов CpG. Справа от нетранслируемого региона (5' UTR) изображено тело гена (Gene body). Синим отмечены метилированные сайты CpG, белым --- неметилированные \autocite{Fu2017}.}\label{fig:CpG_Island}
\end{figure}

Метилирование ДНК может быть обнаружено с помощью различных техник и анализов, таких как масс-спектрометрия, бисульфитное секвенирование генома, анализ кривых плавления с высоким разрешением, технология Illumina \autocite{Rana2018}. Масс-спектрометрия является чувствительным и надёжным методом детектирования метилирования ДНК, однако, этот метод не информативен при исследовании метилирования в контексте последовательности, что является значительным ограничением. Бисульфитное секвенирование генома (BS-Seq) представляет собой высокопроизводительный анализ метилирования в масштабах всего генома. Он основан на конверсии геномной ДНК бисульфитом натрия, которая превращает неметилированные цитозины динуклеотидов CpG в урацил (UpG), а метилированные цитозины при этом не изменяются \autocite{Hernandez2013}. Полученный геном секвенируется на платформах нового поколения, и производится сравнение с исходным геномом для выявления несоответствий. Анализ кривых плавления высокого разрешения основывается на том, что бисульфит-конвертированные высокометилированные последовательности ДНК имеют другую температуру плавления по сравнению с неметилированными \autocite{Malentacchi2009}. Данный метод используется для определения количественной оценки уровня метилирования. Одним из наиболее распространённых методов является технология компании Illumina. Illumina Methylation Assay измеряет метилирование ДНК на каждом локусе с использованием матричной гибридизации. Бисульфит-конвертированная ДНК гибридизируется на специальных чипах «BeadChips». Основными чипами являются 27K Illumina HumanMethylation array и Infinium HumanMethylation450 BeadChip, измеряющие уровень метилирования на 27000 и 450000 сайтах CpG, соответственно. В 2016 году был выпущен чип Infinium MethylationEPIC BeadChip, который измеряет уровень метилирования на 850000 сайтов CpG в геноме человека.

Метилирование ДНК может влиять на транскрипцию генов двумя способами. Во-первых, метилирование ДНК может физически препятствовать связыванию транскрипционных белков с геном \autocite{Choy2010}, во-вторых, сайты CpG могут связываться с белками, известными как метил-CpG-связывающие домены (MBD --- Methyl-CpG-binding domain). Эти белки участвуют в формировании гетерохроматина --- участка хроматина (основы хромосомы) с низкой транскрибируемостью (возможность переноса генетической информации с ДНК на РНК). 

Почти все паттерны метилирования, наследованные от родителей, стираются во время гаметогенеза и в раннем эмбриогенезе. После имплантации эмбриона паттерны метилирования зависят от стадии развития и ткани, а изменения в метилировании каждого отдельного типа клеток стабильно сохраняются на достаточно долгий период \autocite{Cedar2012}. Во время эмбрионального развития немногие гены изменяют свой статус метилирования, за исключением тех генов, которые экспрессируются исключительно у зародышей \autocite{Borgel2010}.

При многих заболеваниях, таких как, например, рак, острова CpG промотора гена становятся аномально гипометилированы, что приводит к подавлению транскрипции \autocite{Wang2018Cancer}. Важным компонентом развития онкологических заболеваний являются изменения в метилировании ДНК. Как правило, при прогрессировании рака активность многих генов либо подавляется, либо активируется. Несмотря на то, что подавление некоторых генов при раке происходит в результате мутаций, большая часть подавленных канцерогенных генов является результатом изменённого метилирования ДНК. 

Такие эпигенетические модификации как метилирование ДНК, также тесно связаны с сердечно-сосудистыми заболеваниями, включая атеросклероз. Гипометилирование ДНК влияет на гены, которые вызывают дисфункцию эндотелиальных клеток (выстилающих внутреннюю поверхность кровеносных и лимфатических сосудов) и увеличивают количество <<медиаторов воспаления>>, что значительным образом влияет на формирование атеросклеротических поражений \autocite{Castro2003}. 

В процессе старения происходит накопление эпигенетических изменений, в частности, глобальное изменение метилирования ДНК. Уровень метилирования различных тканей и органов человека можно использовать для оценки его биологического возраста с помощью, например, эпигенетических часов \autocite{Horvath2013}. 

\subsection{Эпигенетические биомаркеры старения человека}\label{subsec:ch1/sec1/subsec3}

Старение --- сложное ухудшение физиологических процессов с течением времени, происходящее в большинстве живых организмов \autocite{rose1994evolutionary}. Учёные предложили девять признаков старения, которые можно разделить на три категории: первичные, антагонистические и интегративные \autocite{LopezOtin2013}. К первичным признакам относятся нестабильность генома, укорочение теломер, эпигенетические изменения, потеря протеостаза; к антагонистическим --- нарушение распознавания питательных веществ, митохондриальная дисфункция, клеточное старение; к интегративным --- истощение пула стволовых клеток, изменение межклеточного взаимодействия. 

Биомаркеры старения --- это биомаркеры, которые могут прогнозировать функциональную способность отдельных органов и всего организма в целом в более позднем возрасте лучше, чем хронологический возраст \autocite{Baker1988}. Другими словами, биомаркеры старения показывают истинный «биологический возраст», который может отличаться от хронологического возраста. Подтверждённые биомаркеры старения позволят тестировать подходы к увеличению продолжительности жизни. Однако, несмотря на то, что максимальная продолжительность жизни была бы хорошим средством проверки биомаркеров старения, это не подходит для долгоживущих видов, потому что исследования потребовали бы слишком много времени. В идеале биомаркеры старения должны анализировать именно биологический процесс старения, а не предрасположенность к определённым заболеваниям, должны воспроизводиться в течение короткого интервала времени по сравнению с продолжительностью жизни организма.

Распространённое использование данных метилирования ДНК в качестве биомаркеров старения --- построение эпигенетических часов, позволяющий оценить биологический возраст любой ткани организма на протяжении всей жизни \autocite{Horvath2018}. ДНК в лейкоцитах, как правило, гипометилирована с возрастом, в то время как конкретные сайты CpG в промоторах генов имеют тенденцию быть гиперметилированными \autocite{Heyn2012, Gentilini2012}. Существует метод определения биологического возраста, основанный на паттернах метилирования конкретных сайтов CpG, со средней точностью $\pm~3.6$ года с корреляцией $0.96$. Из 353 сайтов CpG 193 оказались гиперметилированными, а 160 --- гипометилированными с возрастом \autocite{Horvath2013}. Связанное с возрастом гиперметилирование преимущественно затрагивает гены, участвующие в контроле метаболизма \autocite{Gentilini2012, Christensen2009}. Напротив, гипометилирование связано с повторяющимися элементами последовательности ДНК, например, Alu-повтор \autocite{Gentilini2012}.

Ряд эпидемиологических исследований показал, что монозиготные близнецы демонстрируют повышенный уровень фенотипического несоответствия, особенно в отношении возрастных заболеваний \autocite{Frederiksen2002, Reynolds2005, Zwijnenburg2010, CastilloFernandez2014}. Это может быть связано с постепенным увеличением скорости изменения метилирования при делении клеток \autocite{Issa2014}. Сравнение старых и молодых монозиготных пар близнецов показывает наличие глобальных различий в паттернах метилирования \autocite{Levesque2014, Tan2016, Wang2018Twins}. Также было показано, что метилом столетнего человека в среднем демонстрирует снижение как уровней метилирования ДНК, так и парных корреляций для соседних CpG сайтов в сравнении с новорождёнными \autocite{Heyn2012}. Мононуклеарные клетки периферической крови итальянских долгожителей (105-109 лет) имели эпигенетический профиль на $8.6$ лет моложе ожидаемого, а их дети (50–89 лет) имели эпигенетический возраст на $5.1$ года меньше, чем контрольная группа того же возраста \autocite{Horvath2015}.

Изменения в метилировании ДНК могут быть обусловлены как генетическими факторами, так и факторами окружающей среды, помимо самого старения. Внешние факторы окружающей среды, такие как, например, курение, пребывание на солнце и ожирение, связаны со специфическими изменениями в паттернах метилирования ДНК \autocite{Gronniger2010, Breitling2011, Almen2014, Vandiver2015}. Важно отметить, что конкретные генетические мутации, встречающиеся с разной частотой или включающие популяционные взаимодействия генов и окружающей среды, могут приводить к этническим различиям в паттернах метилирования ДНК \autocite{Galanter2017, Fagny2015, Heyn2013} и стимулируют различные модели эпигенетического старения. 

Первыми обнаруженными биомаркерами, имеющими высокий уровень корреляции с возрастом, были сайты CpG, располагающиеся в генах ELOVL2, FHL2 и PENK \autocite{Garagnani2012}. В данном исследовании проводился анализ профилей метилирования цельной крови людей в возрасте от 9 до 83 лет (матерей и их детей) с использованием чипа Illumina Infinium HumanMethylation450 BeadChip. Корреляционный анализ Спирмена показал высокие значения коэффициента корреляции между уровнями метилирования и возрастом субъектов. Острова CpG генов ELOVL2, FHL2 и PENK, расположенные в промоторных областях, демонстрируют гиперметилирование с увеличением хронологического возраста, как изображено на рисунке~\cref{fig:ELOVL2}. Впоследствии было показано, что корреляция с возрастом уровня метилирования проб в гене ELOVL2 не является специфичной для цельной крови, а наблюдается также в тканях мозга, кожи, подкожно-жировой клетчатки, печени, щитовидной железы \autocite{Slieker2018}, и, более того, не является специфичной для человека, а наблюдается также, например, у мышей \autocite{Spiers2016}. Тем не менее, большинство коррелирующих с возрастом сайтов CpG являются тканеспецифичными.

\begin{figure}[ht]
	\centerfloat{
		\includegraphics[scale=0.95]{elovl.png}
	}
	\caption{Зависимость уровней метилирования от возраста для сайтов CpG, имеющих наиболее высокое значение коэффициента корреляции. Точками отмечены уровни метилирования матерей, треугольниками --- детей \autocite{Garagnani2012}.}\label{fig:ELOVL2}
\end{figure}

В дополнение к изменению уровня метилирования ДНК с возрастом, существуют также стохастические эпигенетические мутации (эпимутации), которые случайным образом возникают в геноме и не распространяются среди субъектов. Теория соматических мутаций утверждает, что накопление стохастических мутаций в соматических клетках приводит к снижению клеточных функций \autocite{Kennedy2012} и играют важную роль при старении и развитии некоторых возрастных заболеваний. В работе \autocite{Gentilini2015} в популяции нормальных субъектов для каждого сайта CpG цельной крови был оценён нормальный диапазон уровня метилирования. Если значение уровня метилирования у конкретного субъекта выходило за границы нормального диапазона и было чрезвычайно далеко от такового у других субъектов, то в данном CpG сайте присутствует эпимутация. Было показано, что количество стохастических эпигенетических мутаций невелико в детстве и экспоненциально увеличивается с возрастом. 

Однако, в качестве независимой переменной в анализе уровней метилирования ДНК можно рассматривать не только возраст, но и другие характеристики, например, пол, индекс массы тела, наличие вредных привычек и многие другие.

\subsection{Гендерные различия уровней метилирования ДНК}\label{subsec:ch1/sec1/subsec4}

С биологической точки зрения пол у конкретного вида определяется размером гамет: организмы с большими гаметами являются самками, с маленькими --- самцами \autocite{smith1978evolution}. Хорошо известно, что мужчины и женщины отличаются на метаболическом уровне. Например, было обнаружено, что женщины более чувствительны к инсулину, они имеют более низкие уровни липопротеинов низкой плотности в плазме и более высокие уровни липопротеинов высокой плотности по сравнению с мужчинами \autocite{Freedman2004, Magkos2007}. Кроме того, пол может влиять на риск, возраст начала и симптомы заболевания \autocite{Kim2010}. Поэтому важно определить молекулярные механизмы, которые способствуют возникновению этих различий между мужчинами и женщинами. Половые различия связаны не только с половыми хромосомами и гормонами, но и с различиями в экспрессии генов и эпигенетических паттернах \autocite{Mittelstrass2011, Hall2014}. Существуют исследования, рассматривающие гендерные эпигенетические особенности в различных тканях и органах \autocite{Liu2010, Boks2009}. 

Половые различия существуют как в структуре и функциях мозга, так и в степени подверженности неврологическим расстройствам. Например, когнитивные функции, такие как память и внимание, у мужчин и женщин различаются \autocite{Gur1999}. Однако, молекулярная основа различий в восприятии, познании, памяти и нервных функциях мужчин и женщин изучена недостаточно. Есть предположение, что половые различия в функции мозга могут быть связаны с генетическими мутациями \autocite{Qureshi2010} и факторами окружающей среды, такими как пренатальный стресс \autocite{Bowman2004} или гормональный фон \autocite{Lenz2010}.

В работе \autocite{Xu2013} различия в уровне метилирования ДНК префронтальной коры головного мозга были распределены, в основном, по X хромосоме ($92.4~\%$ сайтов CpG, имеющих различный уровень метилирования у мужчин и женщин). Более $65~\%$ сайтов CpG были гиперметилированы у женщин в сравнении с мужчинами. Похожие результаты были получены для пуповинной крови новорождённых с поправкой на клеточный состав \autocite{Yousefi2015}. Гендерные различия в уровне метилирования аутосомной ДНК (не половые хромосомы) стабильны во времени и не зависят от популяционной принадлежности. В образцах цельной крови в генах CISH и RAB23 были найдены различно метилированные сайты CpG у мужчин и женщин, которые приводили к гендерным различиям в экспрессии данных генов \autocite{Singmann2015}. Они демонстрировали более низкие уровни метилирования у женщин по сравнению с мужчинами. Во многих из этих исследований анализировалось только ограниченное количество генов и их участков, таких как промоторная область, и не проводился полногеномный анализ метилирования ДНК.

Анализ гендерных различий уровня метилирования ДНК в масштабах всего генома проводится с помощью Infinium HumanMethylation450 BeadChip. Первое полногеномное исследование сообщило, что существуют как хромосомные, так и специфические для конкретного сайта CpG половые различия в метилировании ДНК на хромосоме X островков поджелудочной железы человека. Однако, аутосомные хромосомы показали гендерные различия в метилировании ДНК только на уровне отдельных CpG-сайтов. Также была обнаружена более высокая секреция инсулина в островках поджелудочной железы у женщин по сравнению с мужчинами и различия в экспрессии генов \autocite{Hall2014}. Половые различия метилирования ДНК клеток печени также были найдены в хромосоме X и аутосомах, как по всей хромосоме, так и по отдельным сайтам CpG. У женщин наблюдалась более высокая экспрессия гена KDM6A и APOA1, подавление которых может способствовать снижению уровня липопротеинов высокой плотности \autocite{GarciaCalzon2018}.

Пол оказывает влияние также на различия в морфологии и метаболизме скелетных мышц \autocite{Haizlip2015, Lundsgaard2014}, на возрастное сокращение мышц и их ремоделирование \autocite{Gheller2016}. Стволовые клетки взрослых скелетных мышц, так называемые сателлитные клетки, отвечают за регенерацию и поддержание скелетных мышц \autocite{Saini2013}. Сателлитные клетки активируются в ответ на стресс, например, после травмы или упражнения. При делении клеток образуются новые стволовые клетки, а также прародители мышечных клеток --- миобласты и миофибры \autocite{Yin2013}. Различия в метилировании ДНК и экспрессии генов между женщинами и мужчинами, которые сохраняются после дифференцировки миобластов в мышечные трубочки, были обнаружены в большей степени на хромосоме X, но также и на аутосомных хромосомах. Кроме того, возникающие в тысячах сайтов CpG половые различия в метилировании ДНК воспроизводились в тканях скелетных мышц в различных популяциях \autocite{Davegardh2019}.

Мужчины и женщины в среднем имеют разные профили риска для различных болезненных состояний, а также разную продолжительность жизни \autocite{RegitzZagrosek2012}. Действительно, средняя ожидаемая продолжительность жизни составляет $69.8$ лет для мужчин и $74.2$ года для женщин, согласно данным Всемирной Организации Здравоохранения на 2016 год \autocite{WHO2016}. Таким образом, особый интерес представляют гендерно-специфичные биомаркеры старения --- биомаркеры, которые с возрастом ведут себя по-разному у мужчин и женщин. Анализ тканеспецифических паттернов экспрессии генов, демонстрирующих зависящие от пола возрастные траектории уровня метилирования ДНК, выявил сильное обогащение в областях семенников и мозга, включая гипоталамус \autocite{McCartney2019}. Это может указывать на связь с эндокринной функцией, что согласуется с гипотезами об эндокринных различиях как об одной из причин неравенства в продолжительности жизни мужчин и женщин \autocite{Austad2016, Ashpole2017}.

\subsection{Однонуклеотидные полиморфизмы ДНК}\label{subsec:ch1/sec1/subsec5}

Одним из основных понятий современной генетики является изменчивость, обусловленная возникновением разных типов мутаций в ДНК живых организмов. Однонуклеотидный полиморфизм (SNP) --- это замена одного нуклеотида в определённом положении в геноме у представителей одного вида, который присутствует в достаточно большой части популяции ($1~\%$ или более). Например, в определённой позиции в геноме человека нуклеотид C может появляться у большинства людей, но у меньшинства людей эту позицию занимает A. Это означает, что в данном конкретном положении есть SNP, и два возможных варианта нуклеотидов --- C или A --- считаются аллелями для этой конкретной позиции. Однонуклеотидные полиморфизмы возникают как результат точечных мутаций. 

SNP могут быть причиной различий в восприимчивости к широкому спектру заболеваний (например, серповидно-клеточная анемия, $\beta$~-талассемия и кистозный фиброз являются результатом SNP) \autocite{INGRAM1956, Chang1979, Reiss1993}. Тяжесть течения болезни и то, как организм реагирует на лечение, также являются проявлением генетических вариаций. Например, однонуклеотидная мутация в гене APOE связана со сниженным риском развития болезни Альцгеймера \autocite{Wolf2013}.

Однонуклеотидные полиморфизмы могут находиться в кодирующих, некодирующих областях генов или в межгенних областях. SNP в кодирующей области бывают двух типов: синонимичные и несинонимичные. Синонимичные полиморфизмы не влияют на аминокислотную последовательность белка, в то время как несинонимичные --- изменяют её. SNP в некодирующих областях могут оказывать влияние на связывание факторов транскрипции, деградацию информационной РНК или последовательность некодирующей РНК. 


С однонуклеотидными полиморфизмами тесно связано неравновесное сцепление генов (LD --- Linkage Disequlibrium) --- неслучайное распределение частот аллелей в двух или более локусах. Считается, что локусы находятся в неравновесном сцеплении, когда частоты их различных аллелей выше или ниже ожидаемых значений, если бы локусы были независимыми и ассоциировались случайным образом \autocite{Slatkin2008}. Неравновесное сцепление может существовать между аллелями в разных локусах без какой-либо генетической связи между ними. 

Предположим, что среди гамет в популяции, размножающейся половым путём, аллель $A$ встречается с частотой $p_{A}$ в некотором локусе, в то время как в другом локусе аллель $B$ встречается с частотой $p_{B}$. Тогда определим $p_{AB}$ как частоту, с которой обе аллели $A$ и $B$ встречаются вместе в одной гамете. Аллели могут рассматриваться как независимые, в таком случае вероятность того, что аллели $A$ и $B$ встретятся вместе, равна произведению вероятностей $p_{A}p_{B}$. Говорят, что существует неравновесное сцепление генов, когда $p_{AB}$ отличается от $p_{A}p_{B}$. Уровень неравновесного сцепления между аллелями $A$ и $B$ определяется следующим образом: 

\begin{equation}
\label{eq:LD}
D_{AB} = p_{AB} - p_{A}p_{B},
\end{equation}
при условии, что оба значения $p_{A}$ и $p_{B}$ больше нуля. Индекс <<$AB$>> подчёркивает, что неравновесное сцепление является свойством пары аллелей ${A, B}$, а не соответствующих им локусов. Другие пары аллелей в тех же двух локусах могут иметь разные коэффициенты неравновесного сцепления.

У людей из разных популяций существует более 335 миллионов SNP \autocite{Auton2015}. Геномное распределение SNP неоднородно --- они встречаются в некодирующих регионах чаще, чем в кодирующих, или, там, где мутация <<зафиксирована>> как наиболее благоприятная генетическая адаптация за счёт естественного отбора \autocite{Barreiro2008}. Между популяциями людей существуют различия, поэтому SNP в одной географической или этнической группе, может встречаться намного реже в другой. Внутри популяции для каждого SNP можно вычислить частоту минорного аллеля --- второго варианта, встречающего в данной популяции \autocite{Zhu2015}. 

Вариации в последовательностях ДНК человека могут влиять на течение болезней и реакцию на патогены, химические вещества, лекарства, вакцины. SNP также имеют важное значение для персонализированной медицины. SNP в клинических исследованиях используется для общегеномного анализа при сравнении определённых областей генома между группами людей (например, с заболеванием и без него) \autocite{Yu2019}. SNP без заметного влияния на фенотип (так называемые <<молчащие мутации>>) могут быть полезны в качестве генетических маркеров в полногеномных ассоциативных исследованиях из-за их стабильной наследственности из поколения в поколение \autocite{Thomas2011}. Некоторые SNP связаны с метаболизмом различных лекарственных препаратов \autocite{Goldstein2001, Yanase2006}. Такие мутации, как удаления, могут замедлять активность ферментов, что может привести к снижению скорости метаболизма лекарственных препаратов \autocite{Butler2018}. Связь широкого спектра заболеваний, таких как рак, инфекционные заболевания (СПИД, гепатит), аутоиммунные, психоневрологические и многие другие, с различными SNP, может быть использована в качестве фармакогеномных мишеней для лекарственной терапии \autocite{Fareed2013}. 

Однако, однонуклеотидные полиморфизмы могут сигнализировать не только о заболеваниях и патологических состояниях. SNP зачастую являются показателями адаптации к определённым климатическим условиям, в которых проживает конкретная популяция, таким как высокогорные регионы с пониженным содержанием кислорода, либо регионы с экстремально низкими температурами. Эти полиморфизмы могут передаваться из поколения в поколение, закрепляясь в генетическом коде.

\subsection{Однонуклеотидные полиморфизмы, связанные с адаптацией к сложным климатическим условиям}\label{subsec:ch1/sec1/subsec6}

Выявление эволюционного изменения генов, связанных с определёнными заболеваниями, иногда даёт информацию об этиологии заболевания. Однако, риск серьёзных заболеваний обычно связан с множественными локусами, а также с факторами окружающей среды. Полногеномные ассоциативные исследования обычно обнаруживают множество геномных маркеров, которые связаны с распространённостью заболевания, либо с увеличением риска заболевания, либо защитой его от него. Почти во всех случаях такие маркеры мало влияют на фенотип \autocite{Blair2015}. Основными факторами, которые могут способствовать распределению геномных полиморфизмов среди популяций, являются миграции за пределы Африки и адаптации к окружающей среде. Карта представлена на рисунке~\cref{fig:map} Миграция может иметь значительное влияние на частоты аллелей; современные люди покинули Африку и начали миграцию через Азию в Америку примерно 45 тыс. лет назад \autocite{Henn2011, Henn2012}. Когда часть популяции мигрирует в новое место, ожидается, что более сильный генетический дрейф в подгруппе приведёт к снижению уровня полиморфной изменчивости. Было показано, что изменения окружающей среды могут быть связаны с паттернами SNP в предположении, что естественный отбор сыграл роль в установлении этих паттернов \autocite{Hancock2011adaptations}. Изменения окружающей среды могут повлиять на частотные характеристики SNP, связанных с заболеванием. Генетический риск некоторых заболеваний изучался в контексте миграции, но в этих исследованиях рассматривались только попарные сравнения популяций. В работе \autocite{Chen2012} объектом изучения были межпопуляционные различия в частотах аллелей риска диабета 2 типа, и оказалось, что этот генетический риск снижается от Африки до Азии и Америки. Это сокращение риска произошло во время миграции за пределы Африки, но было более серьёзным, чем можно было бы ожидать в результате генетического дрейфа. 

\begin{figure}[ht]
	\centerfloat{
		\includegraphics[scale=1.4]{map.jpg}
	}
	\caption{Карта миграции современного человека за последние 100000 лет. На карте показаны события, которые начались с исходной популяции на юге Африки 60-100 тыс. лет назад и завершились заселением Южной Америки 12-14 тыс. лет назад \autocite{Henn2012}.}\label{fig:map}
\end{figure}

С начала 1900-х годов в антропологии и физиологии возник интерес к вопросу изучения генетической адаптации к высоте более 2500 м, поскольку именно там уровни насыщения кислородом в артериальной крови у большинства людей начинают падать \autocite{Julian2019}. Некоторые из первых исследований проводились в Андах, главным образом в Перу и Боливии. Андские жители произошли от первых поселенцев Америки, которые достигли Южной Америки 15--16 тысяч лет назад, а затем разделились на две ветви. Одна из них обосновалась в прибрежных районах Тихого океана, а другая двинулась вдоль побережья Атлантического океана на восток \autocite{GomezCarballa2018}. 

Первоначально адаптация исследовалась в контексте разделения краткосрочных физиологических реакций, называемых акклиматизацией, и реакций, сохраняющихся независимо от продолжительности пребывания на большой высоте. Модель миграции, представленная в 1960-х годах, использовалась, чтобы различать акклиматизацию, и предполагаемые генетические реакции \autocite{Frisancho1970, Frisancho1995, Frisancho2009}. С появлением технологий исследования однонуклеотидных полиморфизмов выяснилось, что жители Анд и другие высокогорные популяции подверглись естественному отбору в нескольких областях генов, влияющих на чувствительные к кислороду пути \autocite{Moore2017}. Ген EGLN1 был идентифицирован как находящийся под воздействием естественного отбора как у жителей Анд, так и жителей Тибета \autocite{Bigham2009}. SNP в данном гене связывают со снижением концентрации гемоглобина в крови жителей высокогорных регионов \autocite{Beall2010}, а промоторная область EGLN1 полностью располагается внутри острова CpG и регулируется эпигенетически в условиях гипоксии \autocite{Lachance2014}.

Для Тибетского нагорья, типичная высота которого более 4000 м, концентрация кислорода составляет всего $60~\%$ от доступной на уровне моря. Было подтверждено, что уникальный набор физиологических адаптаций к хронической гипоксии, наблюдаемый среди жителей Тибета \autocite{hornbein2001high}, имеют генетическую основу \autocite{Simonson2011, Scheinfeldt2013}. В генах с транскрипционными факторами, индуцируемыми гипоксией (HIF --- Hypoxia-Induced Factors) были обнаружены преобразования изменения концентрации кислорода в изменения экспрессии генов \autocite{Kaelin2008, Lendahl2009, Semenza2012}. Это говорит о том, что у коренных высокогорных популяций отбор для адаптации к хронической гипоксии (в отличие от холода, УФ-излучения или какого-либо другого стресса окружающей среды, испытываемого на большой высоте) является ключевым компонентом их недавней эволюции \autocite{Bigham2014}. В нескольких исследованиях использовалось геномное сканирование для отбора среди тибетцев локусов, мутации в которых являются следствием естественного отбора \autocite{Bigham2009, Simonson2010, Yi2010, Wang2011, Xu2010}. Одним из обнаруженных генов, заслуживающих внимания, является HIF2A, который демонстрирует доказательства положительного направленного отбора (отбор по выгодным мутациям, увеличивающим приспособленность носителей) во всех общегеномных анализах \autocite{Bigham2009, Simonson2010, Yi2010, Wang2011, Xu2010}, а SNP в данном гене в значительной степени связаны с низкой концентрацией гемоглобина у тибетцев \autocite{Yi2010}. Второй ген, показывающий доказательства положительного отбора --- это PHD2 \autocite{Simonson2010, Yi2010, Wang2011, Xu2010}. У тибетцев варианты этого гена связаны с концентрацией гемоглобина, тогда как у жителей Анд такой связи не наблюдалось \autocite{Bigham2009, Simonson2010}. 

Микроэволюция человека под действием низких температур также давно привлекает внимание научного сообщества. В настоящее время этому вопросу посвящён ряд исследований как на уровне всего генома \autocite{Hancock2011adaptations, Cardona2014, Valverde2015}, так и на уровне отдельных регионов или генов \autocite{Ohashi2010, Hancock2010population, Sazzini2014, Quagliarello2017}. Одним из наиболее известных генов с точки зрения его роли в адаптации к холодному климату является TRPM8, расположенный на хромосоме 2. Этот ген отвечает за работу ионного канала в качестве теплового датчика, определяющего температуру в диапазоне 15-30~{\textdegree}C \autocite{Fernandez2011}. Его однонуклеотидные полиморфизмы (SNP) также, считается, могут быть связаны с чувствительностью к холоду \autocite{Kozyreva2011}, реакцией дыхательной системы на охлаждение \autocite{Kozyreva2014}, и антропометрическими параметрами \autocite{Potapova2014}. Также были обнаружены доказательства того, что в гене TRPM8 SNP rs10166942 подвергся климатически обусловленному отбору, что привело к увеличению частоты вариаций его аллелей с юга на север \autocite{Key2018}. Ещё 2 SNP, rs17862920 и rs7577262, связаны с ощущением холодовой боли, причём носители аллеля С (распространены в северных широтах) к ней более восприимчивы. Возможным механизмом дифференциальной выживаемости может быть предотвращение потенциально летальной гипотермии у обладателей аллели С \autocite{Igoshin2019}. 

Таким образом, поиск SNP, связанных с адаптацией человека к сложным климатическим условиям, представляет собой актуальную задачу современной популяционной генетики. Раскрытие механизма их действия на гены может предложить новые терапевтические подходы к лечению различных заболеваний, имеющих схожие с адаптацией к сложным условиям симптомы (например, чувствительность к холоду или гипоксия).

\section{Алгоритмы и методы}\label{sec:ch1/sec2}

\subsection{Регрессионный анализ}\label{subsec:ch1/sec2/subsec1}

Регрессионный анализ представляет собой набор процессов для оценки взаимосвязей между зависимой переменной и одной или несколькими независимыми переменными. Регрессионный анализ в основном используется для двух концептуально различных целей. Во-первых, регрессионный анализ широко используется для прогнозирования, где его использование в значительной степени совпадает с областью машинного обучения. Во-вторых, в некоторых ситуациях регрессионный анализ может использоваться для вывода причинно-следственных связей между независимыми и зависимыми переменными. Наиболее распространённой формой регрессионного анализа является линейная регрессия. 



\section{Формулы}\label{sec:ch1/sec3}

Благодаря пакету \textit{icomma}, \LaTeX~одинаково хорошо воспринимает
в~качестве десятичного разделителя и запятую (\(3,1415\)), и точку (\(3.1415\)).

\subsection{Ненумерованные одиночные формулы}\label{subsec:ch1/sec3/sub1}

Вот так может выглядеть формула, которую необходимо вставить в~строку
по~тексту: \(x \approx \sin x\) при \(x \to 0\).

А вот так выглядит ненумерованная отдельностоящая формула c подстрочными
и надстрочными индексами:
\[
(x_1+x_2)^2 = x_1^2 + 2 x_1 x_2 + x_2^2
\]

Формула с неопределенным интегралом:
\[
\int f(\alpha+x)=\sum\beta
\]

При использовании дробей формулы могут получаться очень высокие:
\[
  \frac{1}{\sqrt{2}+
  \displaystyle\frac{1}{\sqrt{2}+
  \displaystyle\frac{1}{\sqrt{2}+\cdots}}}
\]

В формулах можно использовать греческие буквы:
%Все \original... команды заранее, ради этого примера, определены в Dissertation\userstyles.tex
\[
\alpha\beta\gamma\delta\originalepsilon\epsilon\zeta\eta\theta%
\vartheta\iota\kappa\varkappa\lambda\mu\nu\xi\pi\varpi\rho\varrho%
\sigma\varsigma\tau\upsilon\originalphi\phi\chi\psi\omega\Gamma\Delta%
\Theta\Lambda\Xi\Pi\Sigma\Upsilon\Phi\Psi\Omega
\]
\[%https://texfaq.org/FAQ-boldgreek
\boldsymbol{\alpha\beta\gamma\delta\originalepsilon\epsilon\zeta\eta%
\theta\vartheta\iota\kappa\varkappa\lambda\mu\nu\xi\pi\varpi\rho%
\varrho\sigma\varsigma\tau\upsilon\originalphi\phi\chi\psi\omega\Gamma%
\Delta\Theta\Lambda\Xi\Pi\Sigma\Upsilon\Phi\Psi\Omega}
\]

Для добавления формул можно использовать пары \verb+$+\dots\verb+$+ и \verb+$$+\dots\verb+$$+,
но~они считаются устаревшими.
Лучше использовать их функциональные аналоги \verb+\(+\dots\verb+\)+ и \verb+\[+\dots\verb+\]+.

\subsection{Ненумерованные многострочные формулы}\label{subsec:ch1/sec3/sub2}

Вот так можно написать две формулы, не нумеруя их, чтобы знаки <<равно>> были
строго друг под другом:
\begin{align}
  f_W & =  \min \left( 1, \max \left( 0, \frac{W_{soil} / W_{max}}{W_{crit}} \right)  \right), \nonumber \\
  f_T & =  \min \left( 1, \max \left( 0, \frac{T_s / T_{melt}}{T_{crit}} \right)  \right), \nonumber
\end{align}

Выровнять систему ещё и по переменной \( x \) можно, используя окружение
\verb|alignedat| из пакета \verb|amsmath|. Вот так:
\[
    |x| = \left\{
    \begin{alignedat}{2}
        &&x, \quad &\text{eсли } x\geqslant 0 \\
        &-&x, \quad & \text{eсли } x<0
    \end{alignedat}
    \right.
\]
Здесь первый амперсанд (в исходном \LaTeX\ описании формулы) означает
выравнивание по~левому краю, второй "--- по~\( x \), а~третий "--- по~слову
<<если>>. Команда \verb|\quad| делает большой горизонтальный пробел.

Ещё вариант:
\[
    |x|=
    \begin{cases}
    \phantom{-}x, \text{если } x \geqslant 0 \\
    -x, \text{если } x<0
    \end{cases}
\]

Кроме того, для  нумерованных формул \verb|alignedat| делает вертикальное
выравнивание номера формулы по центру формулы. Например, выравнивание
компонент вектора:
\begin{equation}
\label{eq:2p3}
\begin{alignedat}{2}
{\mathbf{N}}_{o1n}^{(j)} = \,{\sin} \phi\,n\!\left(n+1\right)
         {\sin}\theta\,
         \pi_n\!\left({\cos} \theta\right)
         \frac{
               z_n^{(j)}\!\left( \rho \right)
              }{\rho}\,
           &{\boldsymbol{\hat{\mathrm e}}}_{r}\,+   \\
+\,
{\sin} \phi\,
         \tau_n\!\left({\cos} \theta\right)
         \frac{
            \left[\rho z_n^{(j)}\!\left( \rho \right)\right]^{\prime}
              }{\rho}\,
            &{\boldsymbol{\hat{\mathrm e}}}_{\theta}\,+   \\
+\,
{\cos} \phi\,
         \pi_n\!\left({\cos} \theta\right)
         \frac{
            \left[\rho z_n^{(j)}\!\left( \rho \right)\right]^{\prime}
              }{\rho}\,
            &{\boldsymbol{\hat{\mathrm e}}}_{\phi}\:.
\end{alignedat}
\end{equation}

Ещё об отступах. Иногда для лучшей <<читаемости>> формул полезно
немного исправить стандартные интервалы \LaTeX\ с учётом логической
структуры самой формулы. Например в формуле~\cref{eq:2p3} добавлен
небольшой отступ \verb+\,+ между основными сомножителями, ниже
результат применения всех вариантов отступа:
\begin{align*}
\backslash! &\quad f(x) = x^2\! +3x\! +2 \\
  \mbox{по-умолчанию} &\quad f(x) = x^2+3x+2 \\
\backslash, &\quad f(x) = x^2\, +3x\, +2 \\
\backslash{:} &\quad f(x) = x^2\: +3x\: +2 \\
\backslash; &\quad f(x) = x^2\; +3x\; +2 \\
\backslash \mbox{space} &\quad f(x) = x^2\ +3x\ +2 \\
\backslash \mbox{quad} &\quad f(x) = x^2\quad +3x\quad +2 \\
\backslash \mbox{qquad} &\quad f(x) = x^2\qquad +3x\qquad +2
\end{align*}

Можно использовать разные математические алфавиты:
\begin{align}
\mathcal{ABCDEFGHIJKLMNOPQRSTUVWXYZ} \nonumber \\
\mathfrak{ABCDEFGHIJKLMNOPQRSTUVWXYZ} \nonumber \\
\mathbb{ABCDEFGHIJKLMNOPQRSTUVWXYZ} \nonumber
\end{align}

Посмотрим на систему уравнений на примере аттрактора Лоренца:

\[
\left\{
  \begin{array}{rl}
    \dot x = & \sigma (y-x) \\
    \dot y = & x (r - z) - y \\
    \dot z = & xy - bz
  \end{array}
\right.
\]

А для вёрстки матриц удобно использовать многоточия:
\[
\left(
  \begin{array}{ccc}
    a_{11} & \ldots & a_{1n} \\
    \vdots & \ddots & \vdots \\
    a_{n1} & \ldots & a_{nn} \\
  \end{array}
\right)
\]

\subsection{Нумерованные формулы}\label{subsec:ch1/sec3/sub3}

А вот так пишется нумерованная формула:
\begin{equation}
  \label{eq:equation1}
  e = \lim_{n \to \infty} \left( 1+\frac{1}{n} \right) ^n
\end{equation}

Нумерованных формул может быть несколько:
\begin{equation}
  \label{eq:equation2}
  \lim_{n \to \infty} \sum_{k=1}^n \frac{1}{k^2} = \frac{\pi^2}{6}
\end{equation}

Впоследствии на формулы~\cref{eq:equation1, eq:equation2} можно ссылаться.

Сделать так, чтобы номер формулы стоял напротив средней строки, можно,
используя окружение \verb|multlined| (пакет \verb|mathtools|) вместо
\verb|multline| внутри окружения \verb|equation|. Вот так:
\begin{equation} % \tag{S} % tag - вписывает свой текст
  \label{eq:equation3}
    \begin{multlined}
        1+ 2+3+4+5+6+7+\dots + \\
        + 50+51+52+53+54+55+56+57 + \dots + \\
        + 96+97+98+99+100=5050
    \end{multlined}
\end{equation}

Уравнения~\cref{eq:subeq_1,eq:subeq_2} демонстрируют возможности
окружения \verb|\subequations|.
\begin{subequations}
    \label{eq:subeq_1}
    \begin{gather}
        y = x^2 + 1 \label{eq:subeq_1-1} \\
        y = 2 x^2 - x + 1 \label{eq:subeq_1-2}
    \end{gather}
\end{subequations}
Ссылки на отдельные уравнения~\cref{eq:subeq_1-1,eq:subeq_1-2,eq:subeq_2-1}.
\begin{subequations}
    \label{eq:subeq_2}
    \begin{align}
        y &= x^3 + x^2 + x + 1 \label{eq:subeq_2-1} \\
        y &= x^2
    \end{align}
\end{subequations}

\subsection{Форматирование чисел и размерностей величин}\label{sec:units}

Числа форматируются при помощи команды \verb|\num|:
\num{5,3};
\num{2,3e8};
\num{12345,67890};
\num{2,6 d4};
\num{1+-2i};
\num{.3e45};
\num[exponent-base=2]{5 e64};
\num[exponent-base=2,exponent-to-prefix]{5 e64};
\num{1.654 x 2.34 x 3.430}
\num{1 2 x 3 / 4}.
Для написания последовательности чисел можно использовать команды \verb|\numlist| и \verb|\numrange|:
\numlist{10;30;50;70}; \numrange{10}{30}.
Значения углов можно форматировать при помощи команды \verb|\ang|:
\ang{2.67};
\ang{30,3};
\ang{-1;;};
\ang{;-2;};
\ang{;;-3};
\ang{300;10;1}.

Обратите внимание, что ГОСТ запрещает использование знака <<->> для обозначения отрицательных чисел
за исключением формул, таблиц и~рисунков.
Вместо него следует использовать слово <<минус>>.

Размерности можно записывать при помощи команд \verb|\si| и \verb|\SI|:
\si{\farad\squared\lumen\candela};
\si{\joule\per\mole\per\kelvin};
\si[per-mode = symbol-or-fraction]{\joule\per\mole\per\kelvin};
\si{\metre\per\second\squared};
\SI{0.10(5)}{\neper};
\SI{1.2-3i e5}{\joule\per\mole\per\kelvin};
\SIlist{1;2;3;4}{\tesla};
\SIrange{50}{100}{\volt}.
Список единиц измерений приведён в таблицах~\cref{tab:unit:base,
tab:unit:derived,tab:unit:accepted,tab:unit:physical,tab:unit:other}.
Приставки единиц приведены в~таблице~\cref{tab:unit:prefix}.

С дополнительными опциями форматирования можно ознакомиться в~описании пакета \texttt{siunitx};
изменить или добавить единицы измерений можно в~файле \texttt{siunitx.cfg}.

\begin{table}
    \centering
    \captionsetup{justification=centering} % выравнивание подписи по-центру
    \caption{Основные величины СИ}\label{tab:unit:base}
    \begin{tabular}{llc}
        \toprule
        Название  & Команда                & Символ         \\
        \midrule
        Ампер     & \verb|\ampere| & \si{\ampere}   \\
        Кандела   & \verb|\candela| & \si{\candela}  \\
        Кельвин   & \verb|\kelvin| & \si{\kelvin}   \\
        Килограмм & \verb|\kilogram| & \si{\kilogram} \\
        Метр      & \verb|\metre| & \si{\metre}    \\
        Моль      & \verb|\mole| & \si{\mole}     \\
        Секунда   & \verb|\second| & \si{\second}   \\
        \bottomrule
    \end{tabular}
\end{table}

\begin{table}
  \small
  \centering
  \begin{threeparttable}% выравнивание подписи по границам таблицы
    \caption{Производные единицы СИ}\label{tab:unit:derived}
    \begin{tabular}{llc|llc}
        \toprule
        Название       & Команда                 & Символ              & Название & Команда & Символ \\
        \midrule
        Беккерель      & \verb|\becquerel|  & \si{\becquerel}     &
        Ньютон         & \verb|\newton|  & \si{\newton}                                      \\
        Градус Цельсия & \verb|\degreeCelsius| & \si{\degreeCelsius} &
        Ом             & \verb|\ohm| & \si{\ohm}                                         \\
        Кулон          & \verb|\coulomb| & \si{\coulomb}       &
        Паскаль        & \verb|\pascal| & \si{\pascal}                                      \\
        Фарад          & \verb|\farad| & \si{\farad}         &
        Радиан         & \verb|\radian| & \si{\radian}                                      \\
        Грей           & \verb|\gray| & \si{\gray}          &
        Сименс         & \verb|\siemens| & \si{\siemens}                                     \\
        Герц           & \verb|\hertz| & \si{\hertz}         &
        Зиверт         & \verb|\sievert| & \si{\sievert}                                     \\
        Генри          & \verb|\henry| & \si{\henry}         &
        Стерадиан      & \verb|\steradian| & \si{\steradian}                                   \\
        Джоуль         & \verb|\joule| & \si{\joule}         &
        Тесла          & \verb|\tesla| & \si{\tesla}                                       \\
        Катал          & \verb|\katal| & \si{\katal}         &
        Вольт          & \verb|\volt| & \si{\volt}                                        \\
        Люмен          & \verb|\lumen| & \si{\lumen}         &
        Ватт           & \verb|\watt| & \si{\watt}                                        \\
        Люкс           & \verb|\lux| & \si{\lux}           &
        Вебер          & \verb|\weber| & \si{\weber}                                       \\
        \bottomrule
    \end{tabular}
  \end{threeparttable}
\end{table}

\begin{table}
  \centering
  \begin{threeparttable}% выравнивание подписи по границам таблицы
    \caption{Внесистемные единицы}\label{tab:unit:accepted}

    \begin{tabular}{llc}
        \toprule
        Название        & Команда                 & Символ          \\
        \midrule
        День            & \verb|\day| & \si{\day}       \\
        Градус          & \verb|\degree| & \si{\degree}    \\
        Гектар          & \verb|\hectare| & \si{\hectare}   \\
        Час             & \verb|\hour| & \si{\hour}      \\
        Литр            & \verb|\litre| & \si{\litre}     \\
        Угловая минута  & \verb|\arcminute| & \si{\arcminute} \\
        Угловая секунда & \verb|\arcsecond| & \si{\arcsecond} \\ %
        Минута          & \verb|\minute| & \si{\minute}    \\
        Тонна           & \verb|\tonne| & \si{\tonne}     \\
        \bottomrule
    \end{tabular}
  \end{threeparttable}
\end{table}

\begin{table}
    \centering
    \captionsetup{justification=centering}
    \caption{Внесистемные единицы, получаемые из эксперимента}\label{tab:unit:physical}
    \begin{tabular}{llc}
        \toprule
        Название                & Команда                 & Символ                 \\
        \midrule
        Астрономическая единица & \verb|\astronomicalunit| & \si{\astronomicalunit} \\
        Атомная единица массы   & \verb|\atomicmassunit| & \si{\atomicmassunit}   \\
        Боровский радиус        & \verb|\bohr| & \si{\bohr}             \\
        Скорость света          & \verb|\clight| & \si{\clight}           \\
        Дальтон                 & \verb|\dalton| & \si{\dalton}           \\
        Масса электрона         & \verb|\electronmass| & \si{\electronmass}     \\
        Электрон Вольт          & \verb|\electronvolt| & \si{\electronvolt}     \\
        Элементарный заряд      & \verb|\elementarycharge| & \si{\elementarycharge} \\
        Энергия Хартри          & \verb|\hartree| & \si{\hartree}          \\
        Постоянная Планка       & \verb|\planckbar| & \si{\planckbar}        \\
        \bottomrule
    \end{tabular}
\end{table}

\begin{table}
  \centering
  \begin{threeparttable}% выравнивание подписи по границам таблицы
    \caption{Другие внесистемные единицы}\label{tab:unit:other}
    \begin{tabular}{llc}
        \toprule
        Название                  & Команда                 & Символ             \\
        \midrule
        Ангстрем                  & \verb|\angstrom| & \si{\angstrom}     \\
        Бар                       & \verb|\bar| & \si{\bar}          \\
        Барн                      & \verb|\barn| & \si{\barn}         \\
        Бел                       & \verb|\bel| & \si{\bel}          \\
        Децибел                   & \verb|\decibel| & \si{\decibel}      \\
        Узел                      & \verb|\knot| & \si{\knot}         \\
        Миллиметр ртутного столба & \verb|\mmHg| & \si{\mmHg}         \\
        Морская миля              & \verb|\nauticalmile| & \si{\nauticalmile} \\
        Непер                     & \verb|\neper| & \si{\neper}        \\
        \bottomrule
    \end{tabular}
  \end{threeparttable}
\end{table}

\begin{table}
  \small
  \centering
  \begin{threeparttable}% выравнивание подписи по границам таблицы
    \caption{Приставки СИ}\label{tab:unit:prefix}
    \begin{tabular}{llcc|llcc}
        \toprule
        Приставка & Команда                 & Символ      & Степень &
        Приставка & Команда                 & Символ      & Степень   \\
        \midrule
        Иокто     & \verb|\yocto| & \si{\yocto} & -24     &
        Дека      & \verb|\deca| & \si{\deca}  & 1         \\
        Зепто     & \verb|\zepto| & \si{\zepto} & -21     &
        Гекто     & \verb|\hecto| & \si{\hecto} & 2         \\
        Атто      & \verb|\atto| & \si{\atto}  & -18     &
        Кило      & \verb|\kilo| & \si{\kilo}  & 3         \\
        Фемто     & \verb|\femto| & \si{\femto} & -15     &
        Мега      & \verb|\mega| & \si{\mega}  & 6         \\
        Пико      & \verb|\pico| & \si{\pico}  & -12     &
        Гига      & \verb|\giga| & \si{\giga}  & 9         \\
        Нано      & \verb|\nano| & \si{\nano}  & -9      &
        Терра     & \verb|\tera| & \si{\tera}  & 12        \\
        Микро     & \verb|\micro| & \si{\micro} & -6      &
        Пета      & \verb|\peta| & \si{\peta}  & 15        \\
        Милли     & \verb|\milli| & \si{\milli} & -3      &
        Екса      & \verb|\exa| & \si{\exa}   & 18        \\
        Санти     & \verb|\centi| & \si{\centi} & -2      &
        Зетта     & \verb|\zetta| & \si{\zetta} & 21        \\
        Деци      & \verb|\deci| & \si{\deci}  & -1      &
        Иотта     & \verb|\yotta| & \si{\yotta} & 24        \\
        \bottomrule
    \end{tabular}
  \end{threeparttable}
\end{table}

\subsection{Заголовки с формулами: \texorpdfstring{\(a^2 + b^2 = c^2\)}{%
a\texttwosuperior\ + b\texttwosuperior\ = c\texttwosuperior},
\texorpdfstring{\(\left\vert\textrm{{Im}}\Sigma\left(
\protect\varepsilon\right)\right\vert\approx const\)}{|ImΣ (ε)| ≈ const},
\texorpdfstring{\(\sigma_{xx}^{(1)}\)}{σ\_\{xx\}\textasciicircum\{(1)\}}
}\label{subsec:with_math}

Пакет \texttt{hyperref} берёт текст для закладок в pdf-файле из~аргументов
команд типа \verb|\section|, которые могут содержать математические формулы,
а~также изменения цвета текста или шрифта, которые не отображаются в~закладках.
Чтобы использование формул в заголовках не вызывало в~логе компиляции появление
предупреждений типа <<\texttt{Token not allowed in~a~PDF string
(Unicode):(hyperref) removing...}>>, следует использовать конструкцию
\verb|\texorpdfstring{}{}|, где в~первых фигурных скобках указывается
формула, а~во~вторых "--- запись формулы для закладок.

\section{Рецензирование текста}\label{sec:markup}

В шаблоне для диссертации и автореферата заданы команды рецензирования.
Они видны при компиляции шаблона в режиме черновика или при установке
соответствующей настройки (\verb+showmarkup+) в~файле \verb+common/setup.tex+.

Команда \verb+\todo+ отмечает текст красным цветом.
\todo{Например, так.}

Команда \verb+\note+ позволяет выбрать цвет текста.
\note{Чёрный, } \note[red]{красный, } \note[green]{зелёный, }
\note[blue]{синий.} \note[orange]{Обратите внимание на ширину и расстановку
формирующихся пробелов, в~результате приведённой записи (зависит также
от~применяемого компилятора).}

Окружение \verb+commentbox+ также позволяет выбрать цвет.

\begin{commentbox}[red]
        Красный текст.

        Несколько параграфов красного текста.
\end{commentbox}

\begin{commentbox}[blue]
        Синяя формула.

        \begin{equation}
                \alpha + \beta = \gamma
        \end{equation}
\end{commentbox}

\verb+commentbox+ позволяет закомментировать участок кода в~режиме чистовика.
Чтобы убрать кусок кода для всех режимов, можно использовать окружение
\verb+comment+.

\begin{comment}
        Этот текст всегда скрыт.
\end{comment}

\FloatBarrier
