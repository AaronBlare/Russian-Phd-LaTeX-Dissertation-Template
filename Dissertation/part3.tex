\chapter{Метод анализа генетических данных для оценки региональных различий адаптации к климатическим условиям}\label{ch:ch3}

\section{Входные данные}\label{sec:ch3/sec1}

Для анализа генетических данных используется информация об однонуклеотидных полиморфизмах (SNP), распределённых по всему геному человеческого организма. Наиболее обширной базой данных генетических вариантов является проект <<1000 геномов>> \autocite{1000genomes2015, Sudmant2015}, целью которого было найти большинство генетических вариантов с частотой не менее $1~\%$ в изучаемых популяциях и создать Международный ресурс геномных образцов (International Genome Sample Resource --- IGSR). В рамках данного проекта были использованы достижения в области технологии секвенирования, что резко снизило его стоимость. Это был первый проект по секвенированию геномов большого числа людей, чтобы собрать большую базу данных генетических вариаций человека. Секвенирование оставалось слишком дорогим для проведения глубокого анализа многих образцов, изучаемых в рамках проекта. Однако, любая конкретная область генома обычно содержит ограниченное количество гаплотипов. Данные были объединены по образцам, чтобы обеспечить эффективное обнаружение большинства вариантов в изучаемом географическом регионе. В рамках проекта каждый образец секвенировался до 4-кратного покрытия генома; на этой глубине секвенирование не может обнаружить все варианты в каждом образце, но может позволить выявить большинство вариантов с частотами всего $1~\%$. На заключительном этапе проекта были объединены данные из 2504 образцов, чтобы обеспечить высокоточное определение генотипов в каждом образце на всех вариативных участках. Образцы для проекта <<1000 геномов>> анонимны и не имеют связанных медицинских или фенотипических данных. Проект придерживается самооценки этнической принадлежности и пола. Во время сбора образцов все участники заявили, что они здоровы. Текущие образцы проекта <<1000 геномов>> не отражают все популяции (список доступных популяций указан в Таблице~\ref{tab:populations}), однако, его целью является обеспечение максимально возможного разнообразия населения. 

\begin{table} [htbp]
	\centering
	\begin{threeparttable}
		\caption{Популяции, представленные в проекте <<1000 геномов>>}%
		\label{tab:populations}%
		\begin{SingleSpace}
			\begin{tabular}{| c | c | c |}
				\hline
				Код 	& Описание 							& Суперпопуляция \\ \hline
				CHB 	& Хань в Пекине, Китай 				& Восточная Азия \\ \hline
				JPT 	& Японцы в Токио, Япония 			& Восточная Азия \\ \hline
				CHS 	& Хань на юге Китая 				& Восточная Азия \\ \hline
				CDX 	& Дай в Сишуанбаньна, Китай 		& Восточная Азия \\ \hline
				KHV 	& Кин в Хошимине, Вьетнам			& Восточная Азия \\ \hline
				CEU 	& \thead{Жители Юты северного и западноевропейского \\ происхождения} 		& Европа \\ \hline
				TSI 	& Тосканцы в Италии	 				& Европа \\ \hline
				FIN 	& Финны в Финляндии 				& Европа \\ \hline
				GBR 	& Британцы в Англии и Шотландии		& Европа \\ \hline
				IBS 	& Иберийское население в Испании	& Европа \\ \hline
				YRI 	& Йоруба в Ибадане, Нигерия			& Африка \\ \hline
				LWK 	& Лухья в Вебуе, Кения		 		& Африка \\ \hline
				GWD 	& Гамбийцы в западных районах Гамбии										& Африка \\ \hline
				MSL 	& Менде в Сьерра-Леоне				& Африка \\ \hline
				ESN 	& Ишан в Нигерии	 				& Африка \\ \hline
				ASW 	& \thead{Американцы африканского происхождения \\ на юго-западе США}		& Африка \\ \hline
				ACB 	& Вест-индские негры на Барбадосе 	& Африка \\ \hline
				MXL 	& \thead{Мексиканское население Лос-Анджелеса, \\ США} 						& Америка (смешанные) \\ \hline
				PUR 	& Пуэрториканцы в Пуэрто-Рико		& Америка (смешанные) \\ \hline
				CLM 	& Колумбийцы в Медельине, Колумбия 	& Америка (смешанные) \\ \hline
				PEL 	& Перуанцы в Лиме, Перу				& Америка (смешанные) \\ \hline
				GIH 	& Гуджаратцы в Хьюстоне, штат Техас & Южная Азия \\ \hline
				PJL 	& Панджабцы в Лахоре, Пакистан		& Южная Азия \\ \hline
				BEB 	& Бенгальцы в Бангладеше 			& Южная Азия \\ \hline
				STU 	& Ланкийские тамилы	в Великобритании										& Южная Азия \\ \hline
				ITU 	& Индийские телугу в Великобритании	& Южная Азия \\ \hline
			\end{tabular}%
		\end{SingleSpace}
	\end{threeparttable}
\end{table}

Отбор проб для проекта <<1000 геномов>> осуществляется в соответствии со следующими критериями:
\begin{itemize}
	\item Все образцы крови взяты у взрослых субъектов.
	\item Для получения 100 неродственных клеточных линий популяции, первичный материал был собран как минимум от 130 неродственных субъектов.
	\item Получены линии <<бессмертных>> клеток, которые можно использовать для получения практически неограниченных количеств ДНК.
	\item Образцы генотипированы с использованием массива генотипов высокой плотности с более чем 500000 маркеров.
\end{itemize}

Информация об однонуклеотидных полиморфизмах сохраняется в формате VCF (Variant Call Format) --- это формат текстового файла, используемого в биоинформатике для хранения генетических вариаций последовательностей генов. Формат был разработан с появлением крупных проектов секвенирования и фенотипирования ДНК, таких как проект <<1000 геномов>>. Файл формата VCF обязательно включает заголовок, который содержит метаданные, описывающие тело файла. Строки заголовков начинаются с символа $\#$, специальные ключевые слова в заголовке обозначаются $\#\#$. Тело файла представляется собой табулированную таблицу, содержащую 8 обязательных столбцов и неограниченное количество дополнительных столбцов. Когда используются дополнительные столбцы, первый дополнительный столбец используется для описания формата данных в следующих столбцах. Обязательные столбцы:
\begin{enumerate}
	\item CHROM --- имя хромосомы, в которой располагается полиморфизм. 
	\item POS --- номер позиции, в которой находится полиморфизм, отсчитываемый от 1.
	\item ID --- идентификатор полиморфизма, например идентификатор rs или, если он неизвестен, ".".
	\item REF --- референсная аллель в данной позиции.
	\item ALT --- список альтернативных аллелей в данной позиции.
	\item QUAL --- оценка качества, связанная с данными аллелями.
	\item FILTER --- флаг, указывающий, какой из заданного набора фильтров был применён.
	\item INFO --- список пар (полей) ключ-значение, описывающих полиморфизм. 
	\item FORMAT --- (необязательный) расширяемый список полей для описания образцов.
	\item SAMPLES --- значения полей, перечисленных в FORMAT, для каждого (необязательного) образца, описанного в файле.
\end{enumerate}

Для проверки результатов работы предлагаемых алгоритмов использовалась часть данных генетических вариаций проекта <<1000 геномов>>, соответствующих репрезентативным популяциям европейского происхождения из разных широт (Рисунок~\ref{fig:europe_map}) --- GBR (Британцы в Англии и Шотландии), FIN (Финны в Финляндии), TSI (Тосканцы в Италии).

\begin{figure}[ht]
	\centerfloat{
		\includegraphics[scale=0.3]{europe_map.png}
	}
	\caption{Карта Европы с рассматриваемыми популяциями и различными климатическими регионами: субарктический (синий), средиземноморский (оранжевый) и умеренный (зелёный).}\label{fig:europe_map}
\end{figure}

\section{Подход к уменьшению размерности входных данных}\label{sec:ch3/sec2}

Основным объектом исследования являются однонуклеотидные полиморфизмы генома человека, количество которых составляет примерно 5.000.000 в ядерном геноме человека и 3000 --- в митохондриальном. При рассмотрении парных митохондриально-ядерных взаимодействий количество исследуемых пар составляет $1 \cdot 10^{10}$, что затрудняет применение классических статистических методов. Одним из способов снижения размерности является рассмотрение некоторого подмножества ядерных генов с определённой функциональностью, например, связанных с адаптацией к определённым климатическим условиям. В дальнейшем будем рассматривать подмножество из 28 ядерных генов, которые ранее были связаны с адаптацией организма человека к холоду \autocite{Sazzini2014}. Поскольку количество митохондриальных генов невелико (13), и количество однонуклеотидных полиморфизмов в них не превышает 3000, будем рассматривать их полностью. В Таблицах~\ref{tab:snp_mt} и \ref{tab:snp_nuc} представлены рассматриваемые митохондриальные и ядерные гены соответственно, а также количество однонуклеотидных полиморфизмов (SNP) в них.

\begin{table} [htbp]
	\centering
	\begin{threeparttable}
		\caption{Количество однонуклеотидных полиморфизмов (SNP) в митохондриальных генах}%
		\label{tab:snp_mt}%
		\begin{SingleSpace}
			\begin{tabular}{| c | c | c |}
				\hline
				Ген & Хромосома & Количество SNP \\ \hline
				MT-ND3        & MT                  & 77             \\ \hline
				MT-ATP6       & MT                  & 226            \\ \hline
				MT-ATP8       & MT                  & 73             \\ \hline
				MT-CO1        & MT                  & 319            \\ \hline
				MT-CO2        & MT                  & 152            \\ \hline
				MT-CO3        & MT                  & 182            \\ \hline
				MT-CYB        & MT                  & 326            \\ \hline
				MT-ND1        & MT                  & 218            \\ \hline
				MT-ND2        & MT                  & 236            \\ \hline
				MT-ND4        & MT                  & 291            \\ \hline
				MT-ND5        & MT                  & 408            \\ \hline
				MT-ND6        & MT                  & 128            \\ \hline
				MT-RNR1       & MT                  & 118            \\ \hline
				\multicolumn{2}{|l|}{Всего} 		& 2754		     \\ \hline
			\end{tabular}%
		\end{SingleSpace}
	\end{threeparttable}
\end{table}

\begin{table} [htbp]
	\centering
	\begin{threeparttable}
		\caption{Количество однонуклеотидных полиморфизмов (SNP) в рассматриваемых ядерных генах}%
		\label{tab:snp_nuc}%
		\begin{SingleSpace}
			\begin{tabular}{| c | c | c |}
				\hline
				Ген & Хромосома & Количество SNP \\ \hline
				ADRA1A        & 11                  & 7311           \\ \hline
				ADRB3         & 8                   & 203            \\ \hline
				CIDEA         & 18                  & 1445           \\ \hline
				CREB1         & 2                   & 4353           \\ \hline
				DIO2          & 14                  & 10223          \\ \hline
				FTO           & 16                  & 23729          \\ \hline
				HOXC4         & 12                  & 1801           \\ \hline
				HOXA1         & 7                   & 149            \\ \hline
				LIPE          & 19                  & 1485           \\ \hline
				LEP           & 7                   & 863            \\ \hline
				LEPR          & 1                   & 11705          \\ \hline
				NRF1          & 7                   & 6949           \\ \hline
				NRIP1         & 21                  & 753            \\ \hline
				PLIN1         & 15                  & 865            \\ \hline
				PLIN2         & 9                   & 2751           \\ \hline
				PLIN3         & 19                  & 2141           \\ \hline
				PLIN5         & 19                  & 999            \\ \hline
				PPARG         & 3                   & 7539           \\ \hline
				PPARGC1A      & 4                   & 5581           \\ \hline
				PPARGC1B      & 5                   & 7307           \\ \hline
				PRDM16        & 1                   & 28237          \\ \hline
				PRKAR1A       & 17                  & 1869           \\ \hline
				PRKAR2A       & 3                   & 4663           \\ \hline
				PRKAR1B       & 7                   & 14603          \\ \hline
				PRKAR2B       & 7                   & 6691           \\ \hline
				UCP1          & 4                   & 369            \\ \hline
				UCP2          & 11                  & 517            \\ \hline
				UCP3          & 11                  & 587            \\ \hline
				\multicolumn{2}{|l|}{Всего} 		& 155688	     \\ \hline
			\end{tabular}%
		\end{SingleSpace}
	\end{threeparttable}
\end{table}

Рассмотрение всех оставшихся митохондриально-ядерных комбинаций SNP по-прежнему представляет собой задачу огромной размерности. То есть, в лучшем случае митохондриальный ген ATP8 и ядерный ген HOXA1 дадут $73 \times 149 = 10877$ митохондриально-ядерных пар SNP, тогда как MT-ND5 и PRDM16 дадут $408 \times 28237 = 11520696$ митохондриально-ядерных пар SNP, общее количество пар SNP превышает $1 \cdot 10^{8}$. Для того, чтобы уменьшить общее количество рассматриваемых комбинаций SNP, была разработана вычислительная процедура усреднения информации об однонуклеотидных полиморфизмах в рамках каждого рассматриваемого гена.

Рассмотрим три основных типа экспериментов в зависимости от значений, принимаемыми однонуклеотидными полиморфизмами --- митохондриальная ДНК (2 варианта), ядерная ДНК (3 варианта), митохондриально-ядерные взаимодействия ДНК (6 вариантов). Сначала рассмотрим только митохондриальную ДНК, однонуклеотидные полиморфизмы в которой имеют один из двух вариантов --- 0 или 1. Количество вариаций для каждого из рассматриваемых субъектов равно 2754.

Алгоритм уменьшения размерности задачи для митохондриальных генов предполагает выполнение следующих шагов:
\begin{enumerate}
	\item Исключим из рассмотрения такие однонуклеотидные полиморфизмы, которые не меняются среди всех рассматриваемых популяций (имеют значение либо 0, либо 1 для всех субъектов) --- они не несут популяционно-специфической информации.
	\item Зафиксируем одну из рассматриваемых популяций как референсную. Относительно неё будут проводиться дальнейшие вычисления. Конкретный выбор популяции не имеет значения. В данной популяции для каждого из 13 митохондриальных генов вычислим частоту наблюдения каждого из вариантов SNP (0 или 1). Пусть далее $K$ --- количество SNP в гене $Gene N$, а $i = 1, 2, \cdots, K$. Тогда частоты вариантов 0 ($f_{Gene N}^{ref} (0)$) и 1 ($f_{Gene N}^{ref} (0)$) для гена $Gene N$ в референсной популяции будут равны:
	\begin{equation}
	\label{eq:f_ref_mt}
	\begin{gathered}
	f_{Gene N}^{ref} (0) = \frac{k_0}{K},\\
	f_{Gene N}^{ref} (1) = \frac{k_1}{K},
	\end{gathered}
	\end{equation}
	где $k_0$ --- количество вариантов 0 в гене $Gene N$, $k_1$ --- количество вариантов 1 в гене $Gene N$.
	\item Для каждого субъекта во всех рассматриваемых популяциях вычислим частотную метрику каждого гена как среднее расстояние до референсной популяции. Для каждого гена в каждой популяции определим расстояние по средней частоте мутаций как $\left(1 - f_{Gene N}^{ref} (0)\right)$. Если данное расстояние близко к 0, то среднее количество мутаций в данной популяции в данном гене близко к референсной популяции, если расстояние близко к 1 --- среднее количество мутаций в данном гене в данной популяции значительно отличается от референсной популяции. Вычислим средние значение частотной метрики в соответствии с формулой:
	\begin{equation}
	\label{eq:f_mt}
		\begin{gathered}
		f_{Gene N} (0) = \frac{\sum_{SNP_i\,in\,Gene N} \left(1 - f_{Gene N}^{ref} (0)\right)}{K},\\
		f_{Gene N} (1) = \frac{\sum_{SNP_i\,in\,Gene N} \left(1 - f_{Gene N}^{ref} (1)\right)}{K}.
		\end{gathered}
	\end{equation}
	\item Количество вариаций для каждого из рассматриваемых субъектов теперь равно количеству митохондриальных генов --- 13. Для дальнейшего анализа используются полученные значения метрик.
\end{enumerate}

Далее рассмотрим только ядерную ДНК, однонуклеотидные полиморфизмы в которой имеют один из трёх вариантов --- $0|0$, $0|1$ ($1|0$) или $1|1$. Количество вариаций для каждого из рассматриваемых субъектов равно 115688. Алгоритм уменьшения размерности задачи для рассматриваемых ядерных генов предполагает выполнение тех же шагов, за исключением того, что количество вычисляемых частот теперь будет равно 3, а итоговое количество вариаций для каждого из субъектов равно количеству рассматриваемых ядерных генов --- 28.

Рассмотрим наиболее общий случай взаимодействия митохондриальной и ядерной ДНК. При этом пары однонуклеотидных полиморфизмы имеют один из шести парных вариантов, где первый элемент идёт от митохондриальной ДНК, второй идёт от ядерной ДНК --- $0 + 0|0$, $0 + 0|1$ ($0 + 1|0$), $0 + 1|1$, $1 + 0|0$, $1 + 0|1$ ($1 + 1|0$), $1 + 1|1$. Количество пар вариаций для каждого из рассматриваемых субъектов превышает $1 \cdot 10^{8}$.

Алгоритм уменьшения размерности задачи для общего случая взаимодействия митохондриальных и ядерных генов предполагает выполнение следующих шагов:
\begin{enumerate}
	\item Исключим из рассмотрения такие пары однонуклеотидных полиморфизмов, у которых оба элемента не меняются среди всех рассматриваемых популяций.
	\item Зафиксируем одну из рассматриваемых популяций как референсную, относительно которой будут проводиться дальнейшие вычисления. В данной популяции для каждой из пар генов, где для определённости первый ген --- митохондриальный, второй --- ядерный, вычислим частоту наблюдения каждого из вариантов пар SNP. Пусть далее $M$ --- количество митохондриальных SNP в митохондриальном гене $mtGene M$, а $i = 1, 2, \cdots, M$; $N$ --- количество ядерных SNP в ядерном гене $nucGene N$, а $j = 1, 2, \cdots, N$. Тогда частота парного варианта $var$ для пары генов $mtGene M-nucGene N$ в референсной популяции будут равны:
	\begin{equation}
	\label{eq:f_ref_mt_nuc}
	f_{mtGene M-nucGene N}^{ref} (var) = \frac{k_{var}}{M\cdot N},
	\end{equation}
	где $k_var$ --- количество пар варианта $var$ в паре генов $mtGene M-nucGene N$.
	\item Для каждого субъекта во всех рассматриваемых популяциях вычислим частотную метрику каждой пары генов как среднее расстояние до референсной популяции. Для каждой пары генов в каждой популяции определим расстояние по средней частоте мутаций как $\left(1 - f_{mtGene M-nucGene N}^{ref} (var)\right)$. Если данное расстояние близко к 0, то среднее количество мутаций в данной популяции в данной паре генов близко к референсной популяции, если расстояние близко к 1 --- среднее количество мутаций в данной паре генов в данной популяции значительно отличается от референсной популяции. Вычислим средние значение частотной метрики в соответствии с формулой:
	\begin{equation}
	\label{eq:f_mt_nuc}
	\begin{multlined}
	f_{mtGene M-nucGene N} (var) =\\ \frac{\sum_{SNP_i\,in\,mtGene M, SNP_j\,in\,nucGene N} \left(1 - f_{mtGene M-nucGene N}^{ref} (var)\right)}{M\cdot N}.
	\end{multlined}
	\end{equation}
	\item Количество вариаций для каждого из рассматриваемых субъектов теперь равно количеству пар взаимодействий митохондриальных и ядерных генов --- 364. Для дальнейшего анализа используются полученные значения метрик.
\end{enumerate}


