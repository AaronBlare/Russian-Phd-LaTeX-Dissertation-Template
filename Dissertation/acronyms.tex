\chapter*{Список сокращений и условных обозначений} % Заголовок
\addcontentsline{toc}{chapter}{Список сокращений и условных обозначений}  % Добавляем его в оглавление
% при наличии уравнений в левой колонке значение параметра leftmargin приходится подбирать вручную
\begin{description}[align=right,leftmargin=3.5cm]
\item[ДНК] дезоксирибонуклеиновая кислота
\item[мтДНК] митохондриальная дезоксирибонуклеиновая кислота
\item[яДНК] ядерная дезоксирибонуклеиновая кислота
\item[АТФ] аденозинтрифосфат
\item[РНК] рибонуклеиновая кислота
\item[SNP] single nucleotide polymorphism, однонуклеотидный полиморфизм
\item[LD] linkage disequilibrium, неравновесное сцепление генов
\item[OLS] ordinary least squares, метод наименьших квадратов
\item[$\beta$] значение уровня метилирования
\item[aDMP] age-associated differentially methylated probes, связанные с возрастом различно метилированные пробы
\item[sDMP] sex-specific differentially methylated probes, специфичные для пола различно метилированные пробы
\item[saDMP] sex-specific age-associated differentially methylated probes, связанные с возрастом специфичные для пола различно метилированные пробы
\item[saVMP] sex-specific age-associated variable methylated probes, связанные с возрастом специфичные для пола вариабельно метилированные пробы
\end{description}
